%!TEX root = main.tex

% \begin{abstract}

    对于许多应用来说,恢复3D人体姿势是一个具有挑战性的问题。它已经通过带有身体标记和多个摄像头的运动捕捉系统成功解决。在本文中,我们讨论了更具挑战性的案例,不仅使用单个相机,而且还没有利用标记:直接从2D图片到3D几何模型。深度学习方法已经显示出学习2D图片外观特征的显著能力。但是目前该方法缺失的部分是,如何整合2D,3D和时间信息以恢复3D几何结构,并能够考虑由判别模型引起的估计不确定性。我们引入了一种新方法,将2D关节位置视为潜在变量,其不确定性分布由深度卷积神经网络给出。通过稀疏表示对未知3D姿态建模,并且通过期望最大化算法实现3D参数估计,这种方法能够在推理的过程中,方便地使用2D关节位置不确定性。对基准数据集的实验表明,我们所提出的方法比最先进的方法准确性更高。值得注意的是,所提出的方法不需要对应的2D-3D数据来进行训练,并且适用于“野外”图像,我们在MPII数据集上证明了这一点。

% \end{abstract}

