在本节中,介绍了描述3D姿势,2D姿势和图像之间关系的模型。
\subsection{3D姿势的稀疏表示}
3D人体姿势由一组 $p$ 关节的3D位置表示, 第 $t$ 帧记为$\bfS_t\in\RR{3}{p}$ .
一般来说,单视图重建是一个不适定的问题。 为了解决这个问题,假设3D姿势可以表示为预定义基础姿势的线性组合:
\begin{align}\label{eq:shape-model}
    \bfS_t = \sum_{i=1}^{k} c_{it}\bfB_i,
\end{align}
其中$\bfB_i\in\RR{3}{p}$ 表示一组姿势基(basis poses), $c_{it}$ 表示相应的权重. 姿势基(basis poses)是从MoCap数据集提供的训练姿势中学习的。

传统的活动形状模型(active shape model)\cite{cootes1995}中基相对较小,我们没有采用,而是使用稀疏表示的方法,这种方法已经在最近的工作中显示为能够对人体姿势的大变化进行建模,例如\cite{ramakrishna2012reconstructing,akhter2015pose,zhou2015sparse}.

也就是说一个完整(overcomplete)的字典 $\{\bfB_1,\cdots,\bfB_k\}$,是从一个相对较大的姿势集合$k$中学习得到的,其中系数为 $c_{it}$, 假设 $c_{it}$是稀疏的. 在本文的其余部分,$\bfc_t = [c_{1t},\cdots,c_{kt}]^\top$ 表示帧 $t$的系数向量,$\bfC$ 为由所有的 $\bfc_{t}$组成的矩阵。


\subsection{2D和3D姿势之间的依赖性}

\subsubsection{正交相机模型}

当相机固有参数未知时,正交相机模型用于描述3D姿势与其成像2D姿势之间的依赖关系:
\begin{align}\label{eq:camera-model}
    \bfW_t = \bfR_t\bfS_t + \bfT_t\bfone^\top,
\end{align}
其中 $\bfW_t\in\RR{2}{p}$ 表示第$t$帧中的2D姿势, $\bfR_t\in\RR{2}{3}$ 与 $\bfT_t\in\R{2}$依次表示相机的旋转与平移.在后文中, $\bfW$, $\bfR$ 与 $\bfT$ 分别表示所有帧$t$的 $\bfW_t$, $\bfR_t$ , $\bfT_t$ 集合.

考虑到观测噪声和模型误差,给定3D姿态参数,2D姿态的条件分布可以建模为
\begin{align}\label{eq:likelihood}
\pr(\bfW|\theta) \propto e^{-\mathcal{L}(\theta;\bfW)},
\end{align}
其中 $\theta=\{\bfC,\bfR,\bfT\}$ 是所有3D姿势参数的并集。损失函数, $\mathcal{L}(\theta;\bfW)$的定义为
\begin{align}\label{eq:loss}
\mathcal{L}(\theta;\bfW) = \frac{\nu}{2}\sum_{t=1}^{n}\left\|\bfW_t - \bfR_t\sum_{i=1}^{k} c_{it}\bfB_i - \bfT_t\bfone^\top\right\|_F^2,
\end{align}

使用$\|\cdot\|_F$表示Frobenius范数。\refEq{eq:likelihood}中的模型表明,给定3D姿势和相机参数,每个关节的2D位置属于高斯分布,其平均值等于其3D对应物的投影和精度(即,方差的逆)等于$\nu$。

与之前的作品相比\cite{zhou2015sparse},\refEq{eq:loss}中的损失函数等同于先前作品中提出的\cite{zhou2015sparse}在帧上的求和。

%But different from 
我们将模型扩展到透视摄像机的情况,将2D姿势视为潜在变量而不是固定输入,并施加时间平滑约束,如以下小节中所介绍的。

\subsubsection{透视相机模型}

当给出相机固有参数时,由校准矩阵$\bf K$表示,透视相机模型用于描述2D和3D姿势之间的依赖关系:

\begin{align}\label{eq:model-fp}
    \bfK^{-1}\bfW_t\bfZ_t = \bfR_t\bfS_t + \bfT_t\bfone^\top,
\end{align}
其中 $\bfW_t\in\RR{3}{p}$ 表示2D关节的齐次坐标, $\bfR_t\in\RR{3}{3}$ 与 $\bfT_t\in\R{3}$ 表示3D空间中的旋转与平移, $\bfZ_t\in\RR{p}{p}$ 是一个对角矩阵,对角线上的值是关节的深度. 相应地, 透视情况下的损失函数定义为:
\begin{align}\label{eq:loss-fp}\small
\mathcal{L}(\theta;\bfW) = \frac{\nu}{2}\sum_{t=1}^{n}\left\|\bfK^{-1}\bfW_t\bfZ_t - \bfR_t\sum_{i=1}^{k} c_{it}\bfB_i - \bfT_t\bfone^\top\right\|_F^2,
\end{align}
其中 $\theta=\{\bfC,\bfR,\bfT,\bfZ\}$.

注意,最小化\refEq{eq:loss-fp}中的损失会产生一个简单的解,其中所有变量由于透视模型中固有的尺度模糊而收敛到零。 为了避免这种简单的解决方案,通过在优化期间添加以下约束来强制执行根关节的深度:
\begin{align}
z_{1t}=1,
\end{align}
其中 $z_{1t}$ 表示 $\bfZ_t$ 中根节点的对角线上的第一个元素.

\subsection{姿势和图像之间的依赖性}

当给出2D姿势时,我们假设3D姿势参数的分布在条件上独立于图像数据。 因此,$\theta$的似然函数可以分解为
\begin{align}\label{eq:likelihood-complete}
\pr(\bfI,\bfW|\theta) = \pr(\bfI|\bfW)\pr(\bfW|\theta),
\end{align}
其中 $\bfI=\{\bfI_1,\cdots,\bfI_n\}$ 表示图像数据. 条件分布 $\pr(\bfW|\theta)$ 由\refEq{eq:likelihood}. 我们假设由图片数据提供的似然函数 $\pr(\bfI|\bf W)$ 能够通过卷积网络学习得到,并且可以写为
\begin{align}\label{eq:pr-I-W}
\pr(\bfI|\bfW) \propto \Pi_{t}\Pi_{j}h_j(\bfw_{jt};\bfI_t),
\end{align}
其中 $\bfw_{jt}$ 表示关节 $j$在帧 $t$中的位置, $h_j(\cdot;\bfI_t)$ 表示从图片$\bfI_t$ 到关节位置的似然(热图表示)的映射.
对于每个关节 $j$, 这个映射关系 $h_j$ 由卷积网络来进行近似 ( \refSec{sec:cnn} 中进行描述).

\subsection{模型参数的先验}

以下为模型参数的惩罚函数:
\begin{align}\label{eq:prior}
\mathcal{R}(\theta) = \alpha\|\bfC\|_1 + \frac{\beta}{2}\|\nabla_t\bfC\|_F^2 + \frac{\gamma}{2}\|\nabla_t\bfR\|_F^2,
\end{align}
其中$\|\cdot\|_1$ 表示$\ell_1$-norm (即, 绝对值之和), $\nabla_t$ 表示离散的时域差分算符, $\alpha, \beta, \gamma$ 为数值权重. 
第一项惩罚姿势系数的基,使得可以获得稀疏的姿势表示。 第二和第三项对姿势系数和旋转施加一阶平滑约束。
可以对平移分量施加类似的平滑约束; 然而,从实验结果上看,我们没有观察到其明显的性能差异。


