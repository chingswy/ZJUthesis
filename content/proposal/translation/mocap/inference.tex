%!Tex root = main.tex

在本节中,我们描述了我们的三维姿势推理方法。 在这里,区分了两种情况:(i) 提供了关节的图像位置 (\ref{sec:known}节中), 以及
(ii) 关节位置未知 (\ref{sec:unknown}节中).

\subsection{提供了关节的位置}\label{sec:known}

当二维姿势$\bfW$已经给定了,模型参数 $\theta$通过惩罚极大似然估计(MLE)来获得:
\begin{align}\label{eq:pmle}
\theta^* &= \argmax_{\theta} ~ \ln\pr(\bfW|\theta) - \mathcal{R}(\theta) \nonumber \\
& = \argmin_{\theta} ~ \mathcal{L}(\theta;\bfW) + \mathcal{R}(\theta).
\end{align}
 \refEq{eq:pmle} 中的问题通过块坐标下降来解决,即,在固定其他变量时交替更新一个变量块。

\subsubsection{正投影模型}

在正交模型下,损失函数在\refEq{eq:loss}中给出,并且每次迭代中的更新规则在下面给出。

参数 $\bfC$ 的更新需要解:
\begin{align}\label{eq:C-step}
\bfC \leftarrow \argmin_{\bfC} ~~ \mathcal{L}(\bfC;\bfW) + \alpha\|\bfC\|_1 + \frac{\beta}{2}\|\nabla_t\bfC\|_F^2,
\end{align}
其中目标函数是两个可微的函数加上一个$\ell_1$组成。
\refEq{eq:C-step}中的问题通过加速近似梯度解(APG) \cite{nesterov2007gradient}.
由于式\refEq{eq:C-step}中的问题是凸的,因此可以求得全局最优解。 

参数 $\bfR$ 的更新需要解:
\begin{align}\label{eq:R-step}
\bfR \leftarrow \argmin_{\bfR} ~~ \mathcal{L}(\bfR;\bfW) + \frac{\gamma}{2}\|\nabla_t\bfR\|_F^2,
\end{align}
如果目标是可微的,并且变量被旋转限制为$SO(3)$。 这里,采用流形优化的方法,使用Manopt工具箱中的信赖域求解器更新旋转参数\cite{boumal2014manopt}.

参数 $\bfT$ 的更新有以下闭式解:
\begin{align}\label{eq:T-step}
\bfT_t \leftarrow \mbox{row mean}\left\{\bfW_t - \bfR_t\bfS_t\right\}.
\end{align}

\subsubsection{透视相机模型}


在透视模型下,损失函数在\refEq{eq:loss-fp}中给出,并且每次迭代中的更新规则在下面给出。

参数$\bfC$ 和 $\bfR$ 与平行相机投影更新规则相同,式\refEq{eq:C-step} 和 \refEq{eq:R-step}中给出。

参数 $\bfT$ 的更新有以下闭式解:
\begin{align}\label{eq:T-step-fp}
\bfT_t \leftarrow \mbox{row mean}\left\{\bfK^{-1}\bfW_t\bfZ_t - \bfR_t\bfS_t\right\}.
\end{align}

参数 $\bfZ$ 的更新也是闭式的. 假设 $z_{it}$ 代表$\bfZ_t$中第 $i$个对角线元素, $\bfu_{it}$ 表示$\bfK^{-1}\bfW_t$的第$i$列,$\bfv_{it}$表示$\bfR_t\bfS_t+\bfT_t\bfone^\top$的第$i$列。那么每个$z_{it}$ 的解为 
\begin{align}\label{eq:Z-step}
z_{it}\leftarrow 
\begin{cases}
     1 & \mbox{if $i=1$}, \\
    \frac{\bfu_{it}^\top\bfv_{it}}{\bfu_{it}^\top\bfu_{it}} & \mbox{otherwise}
\end{cases}.
\end{align}

给出2D姿势的3D姿态推理的整个算法总结在\refAlg{alg:bcd}中。一旦目标值收敛,迭代就终止。由于在每个步骤中目标函数不增加,因此保证算法终止; 但是,由于\refEq{eq:pmle}中的问题是非凸的,因此该算法需要适当选择的初始化(在\refSec{sec:initialization}中描述)。

\begin{algorithm}[t]\label{alg:bcd}
\LinesNumbered
\caption{在正交投影或透视投影相机模型下,块坐标下降解\refEq{eq:pmle}}
\vspace{0.3em}
\KwIn{$\bfW$ \tcp*[r]{\small 2D关节位置}}
\KwOut{$\theta$ \tcp*[r]{\small 3D姿势参数}}
\vspace{0.3em}
初始化参数 \tcp*{\small \refSec{sec:initialization}}
\While{未收敛}{
update $\bfC$ by \refEq{eq:C-step} with APG \;
update $\bfR$ by \refEq{eq:R-step} with Manopt \;
update $\bfT$ by \refEq{eq:T-step} if 使用正交模型 \\
~~~~~~~~~~~~or by \refEq{eq:T-step-fp} if 使用透视模型 \;
update $\bfZ$ by \refEq{eq:Z-step} if 使用透视模型  \;
}
\end{algorithm}

\subsection{未知的2D姿势}\label{sec:unknown}

如果2D姿势未知,则$\bfW$被视为潜在变量,并在估算过程中被边缘化。 边缘化似然函数是

\begin{align}
\pr(\bfI|\theta) = \int\pr(\bfI,\bfW|\theta)d\bfW, \label{eq:marginalization}
\end{align}
where $\pr(\bfI,\bfW|\theta)$ is given in \refEq{eq:likelihood-complete}.

(\ref{eq:marginalization}) 的直接边缘化是难以处理的。
相反,开发EM算法来计算惩罚的MLE。

在期望步骤中,在给定图像数据和所有3D姿势参数的先前估计的情况下,相对于$\bfW$的条件分布计算惩罚对数似然的期望,$\theta'$
\begin{align}
&Q(\theta|\theta') = \int \left\{\ln\pr(\bfI,\bfW|\theta) - \mathcal{R}(\theta)\right\} ~ \pr(\bfW|\bfI,\theta') d\bfW \nonumber \\
&= \int \left\{\ln\pr(\bfI|\bfW) + \ln\pr(\bfW|\theta) - \mathcal{R}(\theta)\right\} \pr(\bfW|\bfI,\theta') d\bfW \nonumber \\
&= \mbox{const} - \int \mathcal{L}(\theta;\bfW) \pr(\bfW|\bfI,\theta') d\bfW - \mathcal{R}(\theta).
\end{align}

可以证明(参见附录中的推导)
\begin{align}\label{eq:der1}
\int \mathcal{L}(\theta;\bfW) \pr(\bfW|\bfI,\theta') d\bfW = \mathcal{L}(\theta;\mathbb{E}\left[\bfW|\bfI,\theta'\right]) + \mbox{const},
\end{align}
where $\mathbb{E}\left[\bfW|\bfI,\theta'\right]$ is the expectation of $\bfW$ given $\bfI$ and $\theta'$:
\begin{align}\label{eq:der2}
\mathbb{E}\left[\bfW|\bfI,\theta'\right]
&= \int \pr(\bfW|\bfI,\theta')~ \bfW ~ d\bfW \nonumber \\
&= \int \frac{\pr(\bfI|\bfW)\pr(\bfW|\theta')}{M} ~ \bfW ~ d\bfW,
\end{align}
and $M$ is a scalar that normalizes the probability. 
Both $\pr(\bfI|\bfW)$ and $\pr(\bfW|\theta')$ given in \refEq{eq:pr-I-W} and \refEq{eq:likelihood}, respectively, are products of marginal probabilities of $\bfw_{jt}$. Therefore,
the expectation of each $\bfw_{jt}$ can be computed separately. In particular, the expectation of each $\bfw_{jt}$ is efficiently approximated by sampling over the pixel grid.

在最大化步骤中,计算以下内容:
\begin{align}
\theta \leftarrow &\argmax_{\theta} Q(\theta|\theta') \nonumber \\
= &\argmin_{\theta} ~~ \mathcal{L}(\theta;\mathbb{E}\left[\bfW|\bfI,\theta'\right]) + \mathcal{R}(\theta),
\end{align}
which can be solved by \refAlg{alg:bcd}.

The entire EM algorithm is summarized in \refAlg{alg:em} with the initialization scheme described next in \refSec{sec:initialization}.

\begin{algorithm}[t]\label{alg:em}
\LinesNumbered
\caption{用于视频中姿势的EM算法。}
\vspace{0.3em}
\KwIn{$h_j(\cdot;\bfI_t),~ \forall j,t$ \tcp*[r]{\small heat maps}}
\KwOut{$\theta=\{\bfC,\bfR,\bfT,\bfZ\}$ \tcp*[r]{\small pose parameters}}
\vspace{0.3em}
initialize the parameters \tcp*[r]{\small \refSec{sec:initialization}}
\While{not converged}{
$\theta'=\theta$\;
\vspace{0.3em}
\tcp{\small Compute the expectation of $\bfW$}
$\mathbb{E}\left[\bfW|\bfI,\theta'\right] = \int \frac{1}{M}\pr(\bfI|\bfW)\pr(\bfW|\theta') ~ \bfW ~ d\bfW$\;
\vspace{0.3em}
\tcp{\small Update $\theta$ by \refAlg{alg:bcd}}
$\theta = \argmin_{\theta} ~~ \mathcal{L}(\theta;\mathbb{E}\left[\bfW|\bfI,\theta'\right]) + \mathcal{R}(\theta)$ \;
}
\vspace{0.5em}
\end{algorithm}

\subsection{初始化和字典学习}\label{sec:initialization}

A convex relaxation approach \cite{zhou20153d,zhou2015sparse} is used to initialize the parameters.
This convex formulation was initially proposed for pose recovery in a single image frame given 
%to solve the single frame pose estimation problem given 
2D correspondences \cite{zhou20153d}, a special case of \refEq{eq:pmle}.

凸松弛方法 \cite{zhou20153d,zhou2015sparse}用于初始化参数。
最初提出这种凸形配方用于在给定的单个图像帧中进行姿势恢复\cite{zhou20153d}(给定了2D的对应关系),是\refEq{eq:pmle}的一个特例。

该方法后来扩展到处理2D对应异常值\cite{zhou2015sparse}。 如果给出2D姿势,则使用凸松弛分别为每个帧初始化模型参数\cite{zhou2015sparse}。

或者,如果2D姿势未知,则对于每个关节,使用具有最大热图值的图像位置。
接下来,应用来自\cite{zhou2015sparse}的鲁棒估计算法来初始化参数。
字典学习算法\cite{zhou2015sparse}用于学习给定训练姿势数据的姿势字典。基于重建误差和计算效率之间的权衡,字典大小根据经验设置为行动特定字典的$K=64$和非行动特定情况的$K=128$。