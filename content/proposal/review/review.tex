\cleardoublepage
\begin{refsection}
\chapter{文献综述}

\section{背景介绍}
在计算机视觉领域中,人体姿态估计是指利用计算机视觉的技术来估计图像或视频中人体的空间姿态的任务。这个任务在工业环境与生活环境中均有广泛的应用。在人机交互领域,机器人需要与人类进行合作,机器需要感知人体姿态,以实现更有效的交互;在游戏领域,Microsoft Kinect传感器\autocite{shotton2013efficient}使得人类运动姿态捕捉在游戏中的商业应用成为可能;在基于视频的智能监控系统中,对人体的运动姿态的获取能够提供更多的场景中人体的动作及行为信息。

人体姿态估计问题,一般定义为人体关节点定位问题,即对于输入的图像或视频,通过人体姿态估计的方法得到图像或视频中的人体的关节点的位置,从而得到人体相应的姿态。本课题的研究主要关注的问题为人体的三维姿态估计,即根据输入的视频序列,估计出人体关节的空间坐标。


\section{国内外研究现状}
对于三维人体姿态估计问题,目前的方法主要分为两类:端到端法(end-to-end)\autocite{pavlakos2017coarse}和两步法(two-step)\autocite{zhou2016sparseness}。端到端法(end-to-end)\autocite{pavlakos2017coarse}是指对输入的图像或者视频通过一个卷积网络(ConvNet)的处理直接得到人体的三维姿态;而两步法(two-step)\autocite{zhou2016sparseness}通常是指先根据输入的图像或视频,提取出对应的二维人体姿态,再使用优化的方法或神经网络的方法得到人体的三维姿态。

三维人体姿态估计问题,从相机的数量来分又可以分为单目(single-view)三维人体姿态估计和多目(multi-view)三维人体姿态估计。相比于单目的人体姿态估计,多目的方法可以利用多个视角的额外信息,从而从中提取出更为准确的三维人体姿态,消除单个所带来的深度的不确定性。

从估计的人数上来分,三维人体姿态估计又可以分为单人(single-person)三维人体姿态估计和多人(multi-person)三维人体姿态估计,多人的姿态估计问题相比单人的更加复杂,多人之间的人体互相遮挡与交叉的情况更多,因此也是当前研究的热点所在。

\subsection{研究方向及进展}
\subsubsection{二维人体姿态估计}
二维人体姿态估计问题,就是从输入图像或视频中,得到在图片上人体的关节点的位置。传统的人体姿态估计的方法对人体结构进行建模,将人体部位的空间相关性用一个树型结构图表示(tree-structured graphical model)\autocite{eichner20122d}。近些年来,深度卷积网络(ConvNet)在二维人体姿态估计问题中取得了重大的进展。

\subsection{存在问题}

\section{研究展望}

\newpage
\printbibliography
\end{refsection}
