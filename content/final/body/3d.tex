

\newcommand{\qczb}{齐次坐标}
\section{人体三维姿态估计问题的研究}
\subsection{问题描述}
三维人体姿态估计问题,即是说给定输入的人体图片或视频序列,从中得到人体的三维关节点的位置。从上一节中我们已经得到了人体的二维关节点位置,因此对于这一部分我们需要得到相应的人体三维姿态的估计。但是在只有单个视角的情况下,三维估计具有尺度和深度的不确定性。在只有一个相机的情况下,通常使用各种先验信息来减少这种不确定性,在我们的实验场景下,我们可以利用多个相机的空间信息,来优化求解人体的三维关节点的坐标。

\subsection{模型概述}
\subsubsection{相机模型}
相机进行图像拍摄的过程涉及到了多个坐标系的转换,即先从世界坐标系进行刚性变换,转换到相机坐标系,从相机坐标系进行投影变换,转换到图像坐标系,最后根据相机参数,转换到像素坐标系上,像素坐标系就是我们一般意义上所指的图像。几个坐标系的简介如下:
\begin{description}
    \item[世界坐标系:] 是用来描述客观世界的坐标系,描述的是相机三维空间中的其他物体的值,一般用\((x_w, y_w, z_w)\)表示世界坐标系中的值。
    \item[相机坐标系:] 相机坐标系是建立在相机上的坐标系,该坐标系的原点是相机的光心,坐标系的\(x\)轴和\(y\)轴分别平行于图像坐标系的\(x\)轴和\(y\)轴,坐标系的\(z\)轴为相机的光轴。一般用\((x_c,y_c,z_c)\)表示其坐标值。
    \item[图像坐标系:] 图像坐标系的坐标原点是图像平面的中心,图像坐标系是用实际物理单位表示像素在图像中的位置,通常用\((x,y)\)表示其坐标值。
    \item[像素坐标系:] 像素坐标系的原点是图像平面的左上角顶点,通常用\((u,v)\) 表示其坐标值。像素坐标系即是图像在计算机内存储的坐标系,计算机保存时每张图一般用\(M\times N\)的矩阵来表示,其中的每一个元素叫做图像的像素,每一个元素的数值代表图像的像素值。
\end{description}

对于空间中一点,其在世界坐标系中的坐标记为\(P_w\),其在相机坐标系中的坐标记为\(P_c\),那么我们可以使用刚性变换来描述其转换过程,即通过一个旋转矩阵\(R\),与一个平移向量\(T\),将\(P_w\)变换为\(P_c\),即
\begin{equation}
    P_c = RP_w + t
\end{equation}
其中,\(R\)是一个\(3\times 3\)的旋转矩阵,\(T \in \mathbb{R}^{3}\) 表示平移向量。上式在运算的过程中需要进行加法,因此通常在进行计算时,为了简化表示,使用\qczb 来表示点在空间中的位置,从世界坐标系变换到相机坐标系时,其转换方式为
\begin{equation}
\left[\begin{array}{c}X_c\\Y_c\\Z_c\end{array}\right] = \left[\begin{array}{ccc}R_{11}&R_{12}&R_{13}\\R_{21}&R_{22}&R_{23}\\R_{31}&R_{32}&R_{33}\end{array}\right]\left[\begin{array}{c}X_w\\Y_w\\Z_w\end{array}\right] + \left[\begin{array}{c}t_1\\t_2\\t_3\end{array}\right]    
\end{equation}
写成\qczb 的形式为
\begin{equation}
\left[\begin{array}{c}X_c\\Y_c\\Z_c\\1 \end{array}\right] = 
\left[\begin{array}{cccc}R_{11}&R_{12}&R_{13}&t_1\\R_{21}&R_{22}&R_{23}&t_2\\R_{31}&R_{32}&R_{33}&t_3\\0&0&0&1\end{array}\right]
\left[\begin{array}{c}X_w\\Y_w\\Z_w\\1\end{array}\right]
\end{equation}
该转换矩阵就记为相机的外部参数\(T\):
\begin{equation}
    T = \left[\begin{array}{cc}R&T\\0^T&1\end{array}\right]
\end{equation}


\subsubsection{相机标定}
目前主流的相机标定方法是微软研究院的高级研究员张正友教授与1998年提出的一种利用单个棋盘格平面进行相机标定的方法,该方法介于传统标定法和自标定法之间,标定效果较好,并且计算复杂度低。同时不需要高精度的标定物,便于操作。因此我们使用该方法进行相机的标定。

由于LightStage系统的相机较多,与单个相机的标定方法有所区别,因此我们的标定主要分为以下步骤:
\begin{enumerate}
    \item 打印一个棋盘格,已知其实际尺寸
    \item 对每个相机,拍摄不同角度的棋盘格的图片
    \item 对于所有相机,使棋盘格位于所有相机都可见的位置,拍摄系列图片
    \item 检测图片中的棋盘格角点
    \item 对于每个相机,估计其内部参数
    \item 使用所有相机,同时估计所有相机的外部参数
    \item 对参数进行最后的优化
\end{enumerate}

\comment{这里写相机标定的内容}
假设三维世界坐标系中的点的\qczb 为\(X = [x,y,z,1]^T\),其在二维相机平面上的\qczb 为\(m = [u,v,1]^T\)。那么由相机模型的定义我们可以知道:

\begin{equation}
    Z \bm{m} = K(R\bm{X} + T)
\end{equation}
在进行标定时,假设棋盘格所在平面为世界坐标系中的\(Z=0\)的平面,那么则有
\begin{equation}
Z\left[ \begin{array}{c} u \\ v \\ 1 \end{array} \right] = K\left[ 
    \begin{array}{cccc} r_1 & r_2 & r_3 & t\end{array}
    \right]
    \left[ 
    \begin{array}{c}x \\y \\0 \\1\end{array}
    \right]=K
    \left[ 
    \begin{array}{cccc} r_1 & r_2 & T\end{array}
    \right]
    \left[ 
    \begin{array}{c}x \\y \\1\end{array}
    \right]
\end{equation}
其中\(r_1,r_2,r_3\)分别代表旋转矩阵\(R\)的列向量,令
\begin{equation}
    H=[h_1\ h_2\ h3] = \lambda K[r_1\ r_2\ t]
\end{equation}
该矩阵即为棋盘格平面到像素平面的变换矩阵,记为单应性矩阵。\(H\)是一个齐次矩阵,因此有8个未知数,至少需要8个方程来进行求解。由上式可得:
\begin{align}
    \lambda &= \frac{1}{Z} \\
    r_1 &= \frac{1}{\lambda} K^{-1}h_1 \\
    r_2 &= \frac{1}{\lambda} K^{-1}h_2
\end{align}

由于\(R\)是旋转矩阵,\(r_1, r_2\)分别是旋转矩阵的两列,因此这两个向量正交,同时这两个向量都是单位向量,那么则有
\begin{align}
    r_1^Tr_2&=0 \\
||r_1||=||r_2||&=1
\end{align}
代入可得:
\begin{align}
    h_1^TK^{-T}K^{-1}h_2&=0 \\
h_1^TK^{-T}K^{-1}h_1 &= h_2^TK^{-T}K^{-1}h_2
\end{align}
也即是说每个单应性矩阵能够提供两个方程,而内部参数矩阵\(K\)包含了5个参数,因此要进行求解需要至少三个单应性矩阵\(H\)。其中记\(B = A^{-T}A^{-1}\),显然\(B\)为实对称矩阵,将其展开即得
\begin{align}
    B &=K^{-T}K^{-1}=
\left[
\begin{array}{ccc}
B_{11} & B_{12} & B_{13} \\
B_{21} & B_{22} & B_{23} \\
B_{31} & B_{32} & B_{33}
\end{array}
\right] \\ &=
\left[
\begin{array}{ccc}
\dfrac{1}{\alpha^2} & -\dfrac{\gamma}{\alpha^2\beta} & \dfrac{v_0\gamma-u_0\beta}{\alpha^2\beta} \\
-\dfrac{\gamma}{\alpha^2\beta} & \dfrac{\gamma^2}{\alpha^2\beta^2}+\dfrac{1}{\beta^2} & -\dfrac{\gamma(v_0\gamma-u_0\beta)}{\alpha^2\beta^2}-\dfrac{v_0}{\beta^2} \\
\dfrac{v_0\gamma-u_0\beta}{\alpha^2\beta} & -\dfrac{\gamma(v_0\gamma-u_0\beta)}{\alpha^2\beta^2}-\dfrac{v_0}{\beta^2} & \dfrac{(v_0\gamma-u_0\beta)^2}{\alpha^2\beta^2}+\dfrac{v_0}{\beta^2}+1
\end{array}
\right]
\end{align}
由于\(B\)为对称矩阵,因此将\(B\)的有效元素写成向量\(b\),即
\begin{equation}
b=\left[ \begin{array}{cccccc} B_{11} & B_{12} & B_{22} & B_{13} & B_{23} & B_{33} \end{array} \right]^T
\end{equation}
那么之前的式子可以利用向量\(b\)来进行化简:
即式子
\begin{equation}
    h_i^TBh_j = v^T_{ij}b
\end{equation}
其中的系数矩阵为
\begin{equation}
    v_{ij}=\left[ \begin{array}{cccccc} h_{i1}h_{j1} & h_{i1}h_{j2}+h_{i2}h_{j1} & h_{i2}h_{j2} & h_{i3}h_{j1}+h_{i1}h_{j3} & h_{i3}h_{j2}+h_{i2}h_{j3} & h_{i3}h_{j3} \end{array} \right]^T
\end{equation}
那么之前关于矩阵\(B\)的两个约束条件就可以写成
\begin{equation}
    \left[ \begin{array}{c} v^T_{12} \\ (v_{11}-v_{22})^T \end{array} \right]b=0
\end{equation}
因此,基于上式,通过至少三张包含棋盘格的图像,就可以解得矩阵\(B\),再使用Cholesky分解法,就可以得到相机的内部参数矩阵\(K\)。

\comment{相机外参标定}
当所有相机的内部参数确定之后,就需要确定所有相机的外部参数。最开始采用的方法是,对于相邻的两个相机,我们采集一组在这两个相机视角下都可见的棋盘格照片,标定相邻两个相机的相对变换矩阵,接着依次将所有的相机都标定,但是这样标定的过程中,相邻两个相机的标定误差会不断传递,因此整体的标定效果较差。因此,为了避免累积误差的出现,我们直接对所有相机的外参同时进行计算。

\subsubsection{三角法}
% https://scm_mos.gitlab.io/2018/12/08/triangulate/
对于三维空间中的一点$\bfX$,其在世界坐标系中的表示为
\begin{equation}
    \bfX = [x,y,z,1]^T
\end{equation}
第$i$个相机的参数矩阵为
\begin{equation}
    P_i = K_i[R_i | t_i] = \left[ \begin{array}{c}
        P_{i1} \\ P_{i2} \\ P_{i3}
    \end{array}\right]  
\end{equation}
点$X$在第$i$个相机中的投影的坐标为
\begin{equation}
    \bm{x_i} = (x_i, y_i, 1)^T    
\end{equation}
根据投影方程,即有
\begin{equation}
    \bm{x_i} = P_i\bfX
\end{equation}
上式两边同时叉乘$\bm{x_i}$,即有
\begin{equation}
    \bm{x_i} \times (P_i\bfX) = 0
\end{equation}
展开即可得到:
\begin{align}
    x_i(P_{i3}\bm{X}) &- P_{i1}\bm{X} = 0 \\
    y_i(P_{i3}\bm{X}) &- P_{i2}\bm{X}= 0 \\
    x_i(P_{i2}\bm{X}) &- y_iP_{i1}\bm{X}= 0
\end{align}
第三个方程与前两个方程线性相关,因此该组方程实际只提供了两个自由度,即
\begin{align}
    \left[ \begin{array}{c}
        x_iP_{i3} - P_{i1} \\ y_iP_{i3} - P_{i2}
    \end{array}\right]\bm{X} = \bm{0}
\end{align}
因此每个相机上的一个观察点提供了两个约束,而该点的坐标有三个自由度,因此至少需要两个视角才能解出一个点的坐标。而我们的能用的视角个数大于两个,也就是说需要求解超定方程。即求解超静定的齐次线性方程
\begin{equation}
    AX = \bm{0}
\end{equation}
因此我们采用SVD分解的方法,计算$A$矩阵奇异值最小的对应的奇异向量,就是三维坐标的解。那么我们的问题即为假定未知的所有的关节的三维点的坐标为$S\in \mathbb{R}^{3\times N}$,$N$为关节点的数目,第$i$ 个相机的参数为$R_i,T_i,K_i$。那么该问题的方程可写为:
\begin{equation}
    \left[ \begin{array}{c}
        x_{n0}P_{03} - P_{01} \\ 
        y_{n0}P_{03} - P_{02} \\
        x_{n1}P_{13} - P_{11} \\ 
        y_{n1}P_{13} - P_{12} \\
        \vdots \\
        x_{ni}P_{i3} - P_{i1} \\ 
        y_{ni}P_{i3} - P_{i2} 
    \end{array}\right] \left[ \begin{array}{c}
        X_n \\
        Y_n\\
        Z_n\\
        1 
    \end{array}\right]  = \bm{0}, n = 1,2,\ldots,N
\end{equation}
对于所有关节点,求解方程即可获得所有关节点的三维空间坐标。

\subsubsection{优化法}
由于在实际情况中,我们得到的二维关节点不一定在每个相机下都可见,因此我们需要通过一个权值来控制该关节点在一个相机视角下的可见性。同时,由于深度神经网络的输出带有不确定性,其对每个关节的检测结果均会有一个置信度的指标,置信度大即表示深度神经网络模型认为该点为对应关节的概率大,反之置信度小则表示该点是该关节的不确定性较大。那么我们在恢复人体的三维关节坐标时即可将该权重纳入考虑范围内,即更加相信那些置信度高的视角下的检测结果,通过这种方法来使我们的模型更加鲁棒。

由前文可知,相机投影模型为
\begin{equation}
    Z_i \bm{x_i} = K_i(R_i\bm{X} + T_i)
\end{equation}
投影坐标即为
\begin{equation}
    \bm{x_i} = Z_i^{-1}K_i(R_i\bfX + T_i)    
\end{equation}
那么根据各个视角下的观测到的二维坐标,我们的目标函数可以写为
\begin{equation}
    \min \sum^I_{i=1} \sum_{n=1}^N w_{in}||x_{in} - \hat x_{in}||_2
\end{equation}
将投影坐标表达式代入,即得
\begin{equation}
    \min \sum^I_{i=1} \sum_{n=1}^N w_{in}||Z_i^{-1}K_i(R_iX_n + T_i) - \hat x_{in}||_2  
\end{equation}
\newcommand{\mi}{第\(i\)个}
\newcommand{\mn}{第\(n\)个}
其中,$w_{in}$表示\mi 相机视角下,\mn 关节点的对应的深度神经网络检测的二维位置的置信度;$\hat x_{in}$ 表示\mi 相机视角下,\mn 关节点通过深度神经网络估计的在图像平面上的位置;$X_n$为\mn 关节的三维齐次坐标。

\subsection{模型应用}
\comment{写在LightStage数据上采集得到的结果}

\subsection{量化分析}
为了量化评估该方法对
\comment{写在其他数据集上的测试结果,应该用h36m的,这个有marker;或者用cmu panoptic数据集}
