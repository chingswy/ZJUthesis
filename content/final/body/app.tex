\section{人体三维重建工具}

\subsection{总体框架}
整合前面的所有二维关键点估计部分、三维关键点重建、人体几何模型重建部分,我们设计了一个多视角的单目视频人体重建工具。该工具的设计主要流程如下所示。
\unsure{}{插个流程图}

使用者只需要读入多个视角的图像或视频数据,以及摄像机的内部参数和外部参数,就可以对其中的人体动作进行捕捉,并且得到一个人体的三维模型,以及其动作。

\subsection{界面设计}
在界面设计的时候,我们主要秉承着简洁易用的原则,并且能够对我们的结果进行较好的可视化。我们设计的系统的用户交互界面设计如图 所示。该工具包含了人体三维重建的各个步骤,如相机标定、二维关键点估计、三维关键点重建、人体几何模型重建等。

\subsection{使用说明}
在使用时,用户需要依次选择功能。如果相机没有进行标定,需要先选择各个相机标定图片文件所在的文件夹,对其进行标定,得到相机标定的参数。如果相机已经标定,那么需要导入相机参数所在的路径。载入相机参数后,可以选择不同的二维关键点检测方式,进行检测,接着对结果进行可视化。完成二维关键点检测之后,即可运行三维关键点重建,同样提供了两种方式可以选择。最后获得三维关键点之后即可运行人体几何模型重建。

