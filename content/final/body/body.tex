\section{人体二维姿态估计问题的研究}
\subsection{问题描述}
在计算机视觉领域中,人体二维姿态估计是一个极具挑战的任务,人体二维人体姿态估计问题,即是说给定输入的图片或视频序列,进行处理得到相应的人体各个关节在图片上的位置。在深度学习普及之前,该领域的主要研究方法是基于传统的图像处理方法,从图片中提取特征然后进行检测,但是这些传统的处理方法准确率与精度都较低,难以推广到一般场景。自从深度学习方法开始普及之后,对于二维人体姿态估计问题,由于可以由人类标注者在图像中标出人体的关节点的位置,因此可以得到大量的这样的带标注的的训练数据。而基于深度神经网络的方法就可以利用这样大量的数据,对网络进行训练。因此目前的主流的准确率较高的方法都是基于深度学习技术的,本文因此也只介绍这部分内容。

\subsection{研究方法比较}
人体二维姿态估计目前有两个主流方案,第一种是自顶向下的方法,这种方法先检测图片中的人体,得到一个人体的检测框,然后去分别检测每一个人的人体区域中的姿态;另一种是自底向上的方法,首先检测出图片中的所有的人的肢体的关节节点,然后将关节节点进行拼接,得到人体的骨架。自顶向下的方法中,姿态检测的准确度会依赖于人体区域框的检测质量,如果人体区域框的检测不够稳定,那么该方法的输出也会相应的变差。并且如果图片中的人数增加,那么该算法检测到的人体区域的框就会增加,同样的计算时间也会成倍地增加。但是对于自底向上的方法来说,由于是对整张图进行检测,因此计算时间不会随着图片中人数的增加而成倍增加,同时也不会受到前一步检测的人体区域框的质量的影响,检测结果更加鲁棒。但是在自底向上的方法中,如果两个人离得较近,会出现拼接错误的情况;同时由于这种方法依赖的是关节之间的联系,所以对全局信息的获取会有不足。

目前主流的开源人体二维检测方案主要有几种,并且在不同的场景中各有优劣。为了在我们的数据上取得较好的效果,我们对三种方法都进行了研究,并进行了测试,以下依次介绍三种模型。

\subsubsection{OpenPose}
OpenPose\cite{openpose}是由卡耐基梅隆大学的Gines Hidlalgo等人提出的多人二维关键点实时检测方案,该方案是第一个实时的多人二维关键点检测系统,能够同时检测认的身体关键点、手部关键点、脸部关键点及脚部关键点(总共135个关键点)。该方案是基于自底向上的思想的。该方法使用了一种叫部件亲和场(Part Affinity Fields,简称PAF)的非参数表示方法,通过这种方法将检测出来的人体关节组合起来。部件亲和场的表示方法是指,对于人体上的骨架上的每一个像素,去回归这一段骨架的方向向量。如图\ref{fig:2d-op}-(c)所示,对于输入的图片中人体的手臂部分,OpenPose会去回归一个这个手臂上的点沿着人体的关节顺序的指向的方向。例如对于手臂一段,手肘与手腕中间的属于人体的部分的像素,就会输出一个向量表示方向,这些向量都会从手肘指向手腕。该方法具有较高的精度以及较好的实时性。该方法可以同时对人体的身体部分、双脚、手部关节点、脸部关键点进行检测,输出结果较为丰富。
\begin{figure}[H]
    \centering
    \includegraphics[width=\linewidth]{figure/2dpose/openpose}
    \caption{\label{fig:2d-op} OpenPose流程图}
\end{figure}
该方法主要分为几个步骤,其流程图如图\ref{fig:2d-op}所示。首先对于一张输入的尺寸为\(w\times h\)的图片,如图\ref{fig:2d-op}-a,OpenPose将整张图片输入给前馈神经网络,通过该网络预测一系列的二维位置置信图(confidence map)\(\mathbf{S}\),置信图表示的人人体的关节的位置,如图\ref{fig:2d-op}-b所示,以及一系列的二维关节亲和向量场(PAFs)\(\mathbf{L}\),如图\ref{fig:2d-op}-c所示。对于一个人体,如果有\(J\)个关节,那么这一步就会输出\(J\)张置信图,即\(\mathbf{S}=(\mathbf{S}_1, \mathbf{S}_2, ..., \mathbf{S}_J)\)每一张代表了人体的一个关节,其中\(\mathbf{S}_{j}\in \mathbb{R}^{w\times h}$, $j \in \{1\ldots J\}\)。输出的部件亲和场总共有\(C\)个,每一个代表了定义的人体的每一段骨头,即\(\mathbf{L}_{c}\in \mathbb{R}^{w\times h \times 2}$, $c \in \{1\ldots C\}\),其中对于每一段骨头的部件亲和场表示的是一个二维的向量场,\(\mathbf{L}_{c}\in \mathbb{R}^{w\times h \times 2}, c \in \{1\ldots C\}\)。最后通过对回归的结果进行组合,使用贪心策略匹配属于同一个人的关节,就能得到整个人体的完整的身体姿态。如图\ref{fig:2d-op}-e所示。

\begin{figure}[H]
    \centering
    \includegraphics[width=\linewidth]{figure/2dpose/openpose-arch}
    \caption{\label{fig:2d-op-net} OpenPose网络结构}
\end{figure}
OpenPose的网络结构如图\ref{fig:2d-op-net}所示。该网络是一个多阶段的卷积神经网络,图中的蓝色部分负责预测部件亲和场,橙色部分负责预测置信图。由于网络是多阶段的,因此上述两个模块可以不断堆叠在一起,并且对每一个阶段都进行监督,相当于对检测结果不断进行优化。网络中的每个卷积模块使用的是三个堆叠在一起的卷积核为\(3\times 3\)的卷积网络。在输出置信图和部件亲和场后需要对不同的人的身体关节进行组合,由于本文研究的场景都是单人的,因此该部分暂时先不考虑。

\subsubsection{AlphaPose}
AlphaPose\cite{AlphaPose}是上海交通大学卢策吾教授团队开源的二维人体关键点检测方法,该方法使用的是自顶向下的方法,即首先检测人体所在的框的区域,再对框中的人体进行关键点检测。
\begin{figure}[H]
    \centering
    \includegraphics[width=\linewidth]{figure/2dpose/alphapose}
    \caption{\label{fig:2d-ap} AlpahPose流程图}
\end{figure}
该算法流程图如图\ref{fig:2d-ap}所示。首先通过区域候选网络(Region Proposal Network,简写为RPN)提出候选的人体区域,然后将所得到的所有的候选的人体区域框分为两支。图中\ref{fig:2d-ap}上一支通过了空间变换网络(Spatial Transformer Network,简写为STN),将不规则形状的框转化到统一的空间内进行单人的人体姿态估计(Sigle-Person Pose Estimator,简写为SPPE),再将结果通过空间反变换网络(Spatial Detransformer Network,简写为SDTN)转化到原始图片坐标系中。另一支直接通过单人姿态估计(Single Person Pose Estimation)算法,进行姿态估计。最后将两支的结果融合,通过姿态非极大值抑制算法得到姿态。

\subsubsection{级联金字塔网络}
级联金字塔网络(Cascaded Pyramid Network, 简称CPN)\cite{CPN}是由旷视科技提出的模型,该模型取得了COCO2017人体关键点挑战赛的冠军,该模型主要面向的问题是单张图片中的多人姿态估计问题。该模型属于自顶向下的方法,即也是首先生成人体的边界框,之后对单个人体的框进行关键点检测。
\begin{figure}[H]
    \centering
    \includegraphics[width=\linewidth]{figure/2dpose/cpn}
    \caption{\label{fig:2d-cpn} CPN流程图}
\end{figure}
该网络的流程图如图\ref{fig:2d-cpn}所示。级联金字塔网络包含了两个阶段,第一个阶段为全局网络(GlobalNet),这部分用来从图像中提取特征,可以定位特征较为明显的关键点,例如手和脚,但是无法直接从图片中识别出被遮挡的关键点;第二个阶段为精炼网络(RefineNet),这一阶段主要通过整合全局网络中识别得到的图片特征,来推断被遮挡住的关键点的位置。即是说理解可以看到的关键点所提供的上下文信息,完成所有关键点的检测任务。

论文中的全局网络的骨架是深度残差网络\cite{resnet},通过使用U形结构来整合不通尺度的空间分辨率的特征以及语义信息。精炼网络部分中,为了提高计算的效率,同时保持信息在传输过程中的完整性,精炼网络部分首先讲不同尺度的特征图进行堆叠,然后通过上采样和融合,将不同层次的信息进行融合。

\subsection{模型比较}
为了验证上述二维关节点检测模型在我们的实验环境下的准确度,我们需要对其进行定量的分析。
\subsubsection{数据集介绍}
由于我们的实验设备没有在人身上贴标签,因此无法得到准确的三维关键点以及二维关节点的坐标,因此我们选择了广泛使用的Human3.6M数据集\cite{ionescu2014human3}。该数据集是最大的人体三维姿态数据集之一,包含了360万的图片,其中包括了11个演员以及15个日常动作,例如吃饭,走路等。该数据集的采集环境为室内环境,并且使用了总共4个相机来进行数据采集。该数据集通过在人体贴上标志物来获得人体的三维空间的关节位置,相对于从图片直接估计来说较为准确,因此我们通过该数据集来对我们的算法进行定量分析。

\subsubsection{评价标准}
由于我们的任务是进行单人的二维关键点检测,因此不需要考虑多人的二维关键点评价标准。而由这些模型公开的评价指标的数值均是针对多人场景的,因此我们需要对自己得到的结果进行测量。在这里,我们直接使用了在图像中的像素误差值作为真实值与估计值计算的距离,使用的评价标准为平均关节误差,计算公式为
\begin{equation}
    e = \frac{\delta(v_i)\sum_i d_i}{\sum_i\delta(v_i)}
\end{equation}
其中\(d_i\)表示测量的该关节的位置与真实的关节位置的欧式距离,一般写为\(d_i = \sqrt{(\hat{x_i} - x_i)^2 + (\hat{y_i} - y_i)^2}\),这里的\(v_i\)表示第\(i\)个关节的可见性,如果这个关节在图中可见,那么\(v_i = 1, \delta(v_i) = 1\),反之,如果该关节不可见,那么\(v_i = 0, \delta(v_i) = 0\)。这里使用\(\delta\)函数是因为一般的模型输出的是一个介于0-1之间的值,表示其可见性的不确定度。
% 在进行人体二维关键点任务的评价时,一般使用的指标是\textbf{关键点相似度}(Object Keypoint Similarity,简称OKS)。因为对于人体来说,如果简单的使用欧氏距离来进行计算的话,相当于没有考虑到人体在图像中的大小会随着距离的改变而改变。因此该指标会考虑到人体的尺度,计算公式为:
% \begin{equation}
%     OKS_p = \frac{\sum_i }{-d^2_}
% \end{equation}

在这一部分中,我们使用Human3.6M数据集的图片,通过三种不同的人体二维关节点检测方法获得其关节位置,与其提供的二维关节点的位置进行比较,计算三种方法的误差。由于二维关节点的训练数据集中,关节的位置通常都是由人类标注者进行标注,而Human3.6M的关节点的位置是由其贴的标志的位置决定的,因此两者的关节定义有所不同,而在三种二维关节点检测方法中,其各自使用的训练数据集也有所区别,因此我们只选择了人的身体上的主要的几个点进行比较。
\begin{figure}[htbp]
    \centering
    \subfigure[原始图片]{
        \begin{minipage}[t]{0.23\linewidth}
            \centering
            \includegraphics[width=\linewidth]{figure/2dpose/h36m} %
        \end{minipage}% 
    }% 
    \subfigure[OpenPose结果]{
        \begin{minipage}[t]{0.23\linewidth}
            \centering
            \includegraphics[width=\linewidth]{figure/2dpose/h36mop} %
        \end{minipage}% 
    }%
    \subfigure[CPN结果]{
        \begin{minipage}[t]{0.23\linewidth}
            \centering
            \includegraphics[width=\linewidth]{figure/2dpose/h36mcp} %
        \end{minipage}% 
    }%  
    \subfigure[AlphaPose结果]{
        \begin{minipage}[t]{0.23\linewidth}
            \centering
            \includegraphics[width=\linewidth]{figure/2dpose/h36map} %
        \end{minipage}% 
    }%   
    \caption{三种人体二维检测框架在Human3.6M \cite{ionescu2014human}上的结果示例\label{fig:h36mres}}
\end{figure}

\subsubsection{测试结果}
我们取了Human3.6M数据集中的一段视频,该段视频包含了5783帧,时长为1分57秒,视频帧率为50帧每秒。每一个相机拍摄的图片的分辨率为\(1000\times 1002\),与我们的图片的分辨率接近。每一帧有四个视角的相机,即总共有23132张图片,我们对所有的图片都测试了一遍三种模型,得到其各自的结果,运行环境为CPU为 Intel i7-8700,8G RAM,显卡为GeForce GTX 1060 6GB GPU,系统平台为Ubuntu 16.04。OpenPose的源代码为C++实现,需要对源代码进行编译,得到可执行文件运行;AlphaPose基于PyTorch实现,需要配置PyTorch 1.0版本的环境;CPN是基于TensorFlow实现的代码,配置TensorFlow 1.9.0版本即可运行代码。代码配置过程此处省略不表。

三种模型的二维关键点检测结果如图\ref{fig:h36mres}所示。从可视化的结果来看,三种模型的检测结果基本一致,我们需要进一步的定量计算。在这里计算时,由于网络输出具有不确定性,对每个关节的结果都会由一个范围为\((0,1)\)的置信度,我们这里进行量化的时候只考虑那些置信度大于0.5的关节检测结果。首先计算三种方法的检测的平均关节误差,如表\ref{tab:2derror}所示。从表\ref{tab:2derror}中可以看出,AlphaPose的误差均值最低,并且数据的离散程度也较低。OpenPose方法虽然在多人的场景下表现较好,但是在这种单人背景简单的场景下表现不如自顶向下的两种方法。
\begin{table}[H]
    \centering
    \begin{tabular}{lccc}
        \hline
        Name                    & OpenPose & CPN     & AlphaPose \\
        \hline
        \text{误差均值(像素)} & 9.62  & 9.32 & 8.78   \\
        \text{误差标准差}       & 6.35  & 7.85 & 4.63   \\
        \hline
    \end{tabular}
    \caption{三种二维检测人体方法的误差\label{tab:2derror}}
\end{table}

为了更直观的观察其误差,我分别计算了各个关节的误差的分布,并统计了各个关节误差的直方图,如图\ref{fig:2d-loss}所示。从表\ref{tab:2derrorjoint}中可以看出,对于接近人的躯干的部位,如臀部和肩部,其误差的波动大多比处于人的躯干末端的部位小。从误差的直方图中可以看出,对于躯干上的关节点(躯干只左右肩、左右臀部),其误差分布比较接近正态分布,数据都集中在一块,且极差相对较小。而对于属于手和脚的关节点,如图\ref{fig:2d-loss}的第二、三列所示,会出现极个别的误差特别大的点。经过可视化发现,这些误差大的地方主要是人的左右出现了误匹配的情况,因此其置信度较大,但是实际上是匹配错误的,这也是深度神经网络的缺陷之一。因此我们需要在接下来的工作中考虑如何去除掉这些误匹配的离群点,如何利用多视图的信息来对离群点进行剔除。
\begin{table}[H]
    \centering
    \begin{tabular}{lcccccc}
        \hline
        名称      & 左臀     & 左膝    & 左踝     & 右臀     & 右膝    & 右踝     \\
        \hline
        OpenPose  & 11.5,6.0 & 8.8,6.4 & 15.9,7.6 & 15.9,5.0 & 8.0,7.4 & 11.1,9.8 \\
        CPN       & 11.4,5.9 & 7.7,5.5 & 14.0,7.8 & 16.2,5.1 & 6.9,6.6 & 9.8,8.2  \\
        AlphaPose & 12.2,5.2 & 7.8,4.3 & 13.6,5.7 & 17.2,5.2 & 6.7,5.0 & 9.6,5.6  \\
        \hline
        名称      & 左肩     & 左肘    & 左腕     & 右肩     & 右肘    & 右腕     \\
        \hline
        OpenPose  & 7.5,4.7  & 8.0,5.4 & 5.9,6.6  & 9.0,6.4  & 7.3,5.0 & 6.5,6.0  \\
        CPN       & 7.1,4.2  & 7.2,7.4 & 7.1,15.6 & 8.5,5.9  & 7.0,7.8 & 8.9,14.3 \\
        AlphaPose & 6.8,4.3  & 6.7,4.4 & 5.1,3.3  & 7.9,5.4  & 6.0,3.8 & 5.8,3.6  \\
        \hline
        \end{tabular}
    \caption{三种二维检测人体方法的各关节误差均值,方差(单位:像素)\label{tab:2derrorjoint}}
\end{table}


\begin{figure}[htbp]
    \centering
    \includegraphics[width=0.8\linewidth]{figure/2dpose/compare}
    \caption{\label{fig:2d-loss} 各个关节的误差分布对比}
\end{figure}

基于以上比较,我们发现AlphaPose的精度较高,因此我们选择使用该模型在我们的模型上应用。部分图片的二维人体关节点检测结果如图所示。从二维关节点检测结果可以看出,目前的二维人体关节点模型可以在我们的数据上取得较好的结果。基于这一部分的结果,我们才能对人体的三维关节点进行重建。

% \unsure{}{在CMU数据集上也需要测一下,但是他们那个没有2d的GT需要自己投影一下}

\subsection{应用}
在对三种二维检测模型进行比较之后,我们选择了AlphaPose作为我们的二维检测工具,我们采集了三个不同的人的数据,并使用了AlphaPose在数据上进行了应用,得到的部分结果如图\ref{fig:ls2d}所示。图中包含了两个不同的角色的总共四个不同的相机视角。从图中我们可以看出,目前的前沿的二维人体检测方法具有较好的泛化能力,可以在不同的光照条件、人体大小、外观衣着下都能取得好的结果。

\begin{figure}[htbp]
    \centering
    \subfigure[演员冯]{
        \begin{minipage}[t]{0.3\linewidth}
            \centering
            \includegraphics[width=\linewidth]{figure/result/alphapose/vis/feng} %
        \end{minipage}% 
    }% 
    \subfigure[演员帅]{
        \begin{minipage}[t]{0.3\linewidth}
            \centering
            \includegraphics[width=\linewidth]{figure/result/alphapose/vis/shuai} %
        \end{minipage}% 
    }%
    \subfigure[演员柯]{
        \begin{minipage}[t]{0.3\linewidth}
            \centering
            \includegraphics[width=\linewidth]{figure/result/alphapose/vis/ke} %
        \end{minipage}% 
    }%  
    \caption{人体二维检测框架AlphaPose在LightStage数据上的应用结果\label{fig:ls2d}}
\end{figure}

尽管OpenPose在精度上低于AlphaPose,但是该模型能够同时检测图片中的手部关键点、脸部关键点,以及脚部关键点。因此我们同样使用该模型在我们的数据上进行了测试,并对结果进行了可视化。实验环境为Ubuntu 16.04, GPU GeForce GTX 1060 6GB,平均运行时间为2.59帧每秒。

\begin{figure}[htbp]
    \centering
    \subfigure[演员冯]{
        \begin{minipage}[t]{0.3\linewidth}
            \centering
            \includegraphics[width=\linewidth]{figure/result/openpose/feng_rendered} %
        \end{minipage}% 
    }% 
    \subfigure[演员帅]{
        \begin{minipage}[t]{0.3\linewidth}
            \centering
            \includegraphics[width=\linewidth]{figure/result/openpose/shuai_rendered} %
        \end{minipage}% 
    }%
    \subfigure[演员柯]{
        \begin{minipage}[t]{0.3\linewidth}
            \centering
            \includegraphics[width=\linewidth]{figure/result/openpose/ke_rendered} %
        \end{minipage}% 
    }%  
    \caption{人体二维检测框架OpenPose在LightStage数据上的应用结果\label{fig:ls2dop}}
\end{figure}
% \unsure{}{感觉需要一下将图片做个直方图矫正}
从可视化的结果(图\ref{fig:ls2d},图\ref{fig:ls2dop})中我们可以看到,基于深度学习的方法能够在我们的场景中容易地检测出人体的各个关节的位置,我们的演员的服装、动作均可以不受实验环境的限制,可以较为自由地实现各种动作,并且二维检测方法还能检测出这些动作下的关键点位置。同时,由于我们的视频采集得到的是分辨率为\(2048\times 2048\)的高清照片,因此即使在这种有一定距离的场景下,依然能保持脸部的清晰度,也能够实现对人体面部关键点、手部关键点的较为准确的检测。

\subsection{本章小结}
在本章中,我们对三种基于深度学习的人体二维关键点检测方法进行了学习,实现了在我们的数据上对人体二维关键点的检测功能。为了对二维关键点检测进行比较,我们选择了与我们的实验场景类似的人体大型开源数据集Human3.6M,并在该数据集上进行了定量的实验与误差分析,得到了AlphaPose精度更高的结论,为后续的工作提供了第一步的结果。这一部分存在的问题是,基于深度学习的方法在进行人体二维关键点检测时会出现检测错误的情况,因此我们在接下来的工作中会继续探究这一问题。



\section{人体三维姿态估计问题的研究}
\subsection{问题描述}
三维人体姿态估计问题,即是说给定输入的人体图片或视频序列,从中得到人体的三维关节点的位置。从上一节中我们已经得到了人体的二维关节点位置,因此对于这一部分我们需要得到相应的人体三维姿态的估计。但是在只有单个视角的情况下,三维估计具有尺度和深度的不确定性。在只有一个相机的情况下,通常使用各种先验信息来减少这种不确定性,在我们的实验场景下,我们可以利用多个相机的空间信息,来优化求解人体的三维关节点的坐标。

\subsection{模型概述}
\subsubsection{相机模型}
\comment{先写一下相机模型的定义}

\comment{这里写相机标定的内容}
\subsubsection{三角法}
% https://scm_mos.gitlab.io/2018/12/08/triangulate/
对于三维空间中的一点$\bfX$,其在世界坐标系中的表示为
\begin{equation}
    \bfX = [x,y,z,1]^T
\end{equation}
第$i$个相机的参数矩阵为
\begin{equation}
    P_i = K_i[R_i | t_i] = \left[ \begin{array}{c}
        P_{i1} \\ P_{i2} \\ P_{i3}
    \end{array}\right]  
\end{equation}
点$X$在第$i$个相机中的投影的坐标为
\begin{equation}
    \bm{x_i} = (x_i, y_i, 1)^T    
\end{equation}
根据投影方程,即有
\begin{equation}
    \bm{x_i} = P_i\bfX
\end{equation}
上式两边同时叉乘$\bm{x_i}$,即有
\begin{equation}
    \bm{x_i} \times (P_i\bfX) = 0
\end{equation}
展开即可得到:
\begin{align}
    x_i(P_{i3}\bm{X}) &- P_{i1}\bm{X} = 0 \\
    y_i(P_{i3}\bm{X}) &- P_{i2}\bm{X}= 0 \\
    x_i(P_{i2}\bm{X}) &- y_iP_{i1}\bm{X}= 0
\end{align}
第三个方程与前两个方程线性相关,因此该组方程实际只提供了两个自由度,即
\begin{align}
    \left[ \begin{array}{c}
        x_iP_{i3} - P_{i1} \\ y_iP_{i3} - P_{i2}
    \end{array}\right]\bm{X} = \bm{0}
\end{align}
因此每个相机上的一个观察点提供了两个约束,而该点的坐标有三个自由度,因此至少需要两个视角才能解出一个点的坐标。而我们的能用的视角个数大于两个,也就是说需要求解超定方程。即求解超静定的齐次线性方程
\begin{equation}
    AX = \bm{0}
\end{equation}
因此我们采用SVD分解的方法,计算$A$矩阵奇异值最小的对应的奇异向量,就是三维坐标的解。那么我们的问题即为假定未知的所有的关节的三维点的坐标为$S\in \mathbb{R}^{3\times N}$,$N$为关节点的数目,第$i$ 个相机的参数为$R_i,T_i,K_i$。那么该问题的方程可写为:
\begin{equation}
    \left[ \begin{array}{cccc}
        x_{00}P_{03} - P_{01}  & x_{10}P_{03} - P_{01} & \cdots & x_{n0}P_{03} - P_{01} \\ 
        y_{00}P_{03} - P_{02}  & y_{10}P_{03} - P_{02} & \cdots & y_{n0}P_{03} - P_{02} \\
        x_{01}P_{13} - P_{11}  & x_{11}P_{13} - P_{11} & \cdots & x_{n1}P_{13} - P_{11} \\ 
        y_{01}P_{13} - P_{12}  & y_{11}P_{13} - P_{12} & \cdots & y_{n1}P_{13} - P_{12} \\
        \vdots & \vdots & \ddots & \vdots \\
        x_{0i}P_{i3} - P_{i1}  & x_{1i}P_{i3} - P_{i1} & \cdots & x_{ni}P_{i3} - P_{i1} \\ 
        y_{0i}P_{i3} - P_{i2}  & y_{1i}P_{i3} - P_{i2} & \cdots & y_{ni}P_{i3} - P_{i2}
    \end{array}\right] \left[ \begin{array}{cccc}
        X_0 & X_1 & \cdots & X_n \\
        Y_0 & Y_1 & \cdots & Y_n \\
        Z_0 & Z_1 & \cdots & Z_n \\
        1 & 1 & \cdots & 1
    \end{array}\right]  = \bm{0}
\end{equation}


\subsubsection{优化法}
相机投影模型为
\begin{equation}
    Z_i W_i = K_i(R_iS + T_i)
\end{equation}
\begin{equation}
W_i = Z_i^{-1}K_i(R_iS + T_i)    
\end{equation}
其中$W_i\in \mathbb{R}^{3\times p}$  表示第$i$ 个相机下的二维齐次坐标,那么根据各个视角下的观测到的二维坐标,我们的目标函数可以写为
\begin{equation}
    \min \sum^I_{i=1} \sum_{p=1}^P \omega_{ip}||W_{ip} - \hat W_{ip}||_2
\end{equation}
\begin{equation}
    \min \sum^I_{i=1} \sum_{p=1}^P \omega_{ip}||Z_i^{-1}K_i(R_iS_p + T_i) - \hat W_{ip}||_2  
\end{equation}

\subsection{模型应用}
\comment{写在LightStage数据上采集得到的结果}


\section{人体几何模型的应用}
\subsection{问题描述}
在上一部分中,我们已经从图片中得到人体的三维关节点的坐标,但是在实际应用中,简单的骨架表示不够具有表现力,因此最新的论文里的做法是使用一个预先生成好的参数化的人体几何模型,通过优化算法,或者直接通过网络估计,得到该模型的参数。在我们的问题中,希望通过输入一个人体的骨架的三维坐标,输出该参数化模型的参数。由于骨架坐标无法完全确定该模型的参数,因此我们需要引入一些额外的约束,来完成该求解过程。

\subsection{模型介绍}
SMPL模型是一个高效的线性人体模型,模型总共定义了\(N = 6890\)个顶点,作为人体的模板\(\bar{\bm{T}}\)。为了使该模型能够描述不同的人体姿态,该模型定义了人体的24个关节,每个关节具有三个旋转的自由度(包括整个身体的旋转),再加上身体的平移的三个自由度,该模型使用了\(3\times 24 + 3 = 75\)个位姿参数\(\theta\)来描述人体的姿态。为了能够描述不同人体的形状的差异,模型使用了形状参数\(\beta\)来对人体模板的点进行非刚性变形。也就是说,人体模板首先会根据形状参数和姿态参数进行变形:
\begin{equation}
    T(\beta, \theta) = \bar{\bm{T}} + B_S(\beta) + B_P(\theta)
\end{equation}
这里的\(B_S(\beta)\)是一个形状混合函数,他将形状参数\(\beta \in \mathbb{R}^{|\beta|}\)映射到人的模型的顶点所在的空间内,即\(B_S(\beta):\mathbb{R}^{|\beta|} \mapsto \mathbb{R}^{3N}\)。这里的\(B_P(\beta)\)是一个依赖于姿势的形状混合函数,他将姿势参数\(\theta \in \mathbb{R}^{|\theta|}\)同样映射到人的模型的顶点所在的空间内,即\(B_P(\beta):\mathbb{R}^{|\theta|} \mapsto \mathbb{R}^{3N}\),这一项考虑的是人体在进行不同的动作的时候,对身体的造成的形变。这两项形状混合函数都是直接将点的坐标加到人体的静止姿态下,也就是说先对人体的静止姿态进行变形,完成对形状的改变。在这之后,再对人体模型上的所有点进行混合蒙皮操作(blend skinning),将非关键点的位置进行旋转平移,得到经过姿势变换后的人体模型上的点的位置。

\textbf{混合蒙皮(Blend skinning):}这一步的目的是为了根据输入的姿态参数\(theta\)将人体的网格模型进行变形。人体的姿态参数的定义是,对于每个关节,使用轴角(axis-angle)来表示他的旋转。由于每一个旋转矩阵只有三个自由度,所以一般的表示方法都是使用参数化的表示。常用的参数化的方法是使用欧拉角,但是欧拉角会带来万向锁的问题,因为欧拉角表示并没有覆盖旋转矩阵空间内的所有区域。并且,欧拉角使用角度来进行表示,这样在进行优化的时候其值的范围是有周期性的,不连续,使得优化过程不自然,因此选择了使用轴角表示。

对于我们的骨架模型,有\(K=23\)个关节表示旋转,加上人整体的旋转,姿势参数即为\(\theta = [\omega_0^T, \omega_1^T, \ldots, \omega_K^T]\),其中的每一个\(\omega\)是一个三维的向量,\(\omega\)的模长\(||\omega||\)表示旋转的角度,\(\frac{\omega}{||\omega||}\)表示旋转的方向的单位向量。为了在计算的时候表示旋转,通常会需要先转换成旋转矩阵。旋转向量转化成旋转矩阵的方法是通过罗德里格斯公式(Rodrigues formula):

\begin{equation}
    \exp \left(  \omega  _ { j } \right) = \mathcal { I } + \widehat { \overline { \omega }} _ { j } \sin \left( \left\|  \omega  _ { j } \right\| \right) + \widehat { \overline { \omega } } _ { j } ^ { 2 } \cos \left( \left\| \omega_ { j } \right\| \right)
\end{equation}
其中\(\overline { \omega } = \frac{\omega}{||\omega||}\)表示旋转轴的单位向量,\(\widehat { \overline { \omega }}\)表示关于三维向量\(\overline{\omega}\)的反对称矩阵,即对于向量\(\mathbf{K} = [k_x, k_y, k_z]^T\),其计算公式为
\begin{equation}
\widehat{\mathbf { K }} = \left[ \begin{array} { c c c } { 0 } & { - k _ { z } } & { k _ { y } } \\ { k _ { z } } & { 0 } & { - k _ { x } } \\ { - k _ { y } } & { k _ { x } } & { 0 } \end{array} \right]
\end{equation}
\(\mathcal { I }\)表示\(3\times 3\)的单位矩阵。那么通过这一步计算,我们即将所有的\(K \times 3\)的旋转向量转换成了\(K\times 3 \times 3\)的旋转矩阵。由于在进行点的坐标计算的时候,一般都使用齐次坐标来表示点的位置,因此我们需要将旋转矩阵扩充为\(4\times 4\)的齐次变换矩阵。也即是说对于每个关节,都有一个局部的齐次变换矩阵,写为
\begin{equation}
\left[ \begin{array} { c | c } { \exp \left( \vec { \omega } _ { j } \right) } & { \mathbf { j } _ { j } } \\ \hline \overrightarrow { 0 } & { 1 } \end{array} \right]
\end{equation}
计算该关节相对于全局坐标系的齐次变换矩阵时,即需要根据骨架连接的顺序,依次将各个关节的局部变换矩阵相乘,即可得到该关节的全局变换矩阵,即
\begin{equation}
    G _ { k } ( \vec { \theta } , \mathbf { J } ) = \prod _ { j \in A ( k ) } \left[ \begin{array} { c | c } { \exp \left( \vec { \omega } _ { j } \right) } & { \mathbf { j } _ { j } } \\ \hline \overrightarrow { 0 } & { 1 } \end{array} \right]
\end{equation}
其中,\(A(k)\)表示对于关节\(k\),其所有父节点的按连接顺序排列的有序集合。\comment{插个图片,距离}。该式子表示,对于关节\(k\),其全局变换矩阵的计算为其父节点的全局变换矩阵的依次矩阵相乘。\(\mathbf{j}_j\)表示关节\(j\)的相对于其父节点的位移,其长度即为该段骨长。
 
\subsection{算法分析}
\comment{这里写如何Fit,就是简单的loss函数}

我们都知道,人体是具有骨架结构的,也就是说人的运动状态一定是连续的,人的关节坐标无法突变,因此,为了增加结果的鲁棒性,以及重建的人的运动的平滑性,我们通过增加时序误差来对求解结果进行优化。时序误差的定义为

\begin{align} 
    E _ { T } ( \beta , \theta , \mathbf { \Omega } ) = \sum _ { t = 2 } ^ { T } \sum _ { i = 1 } ^ { J } \lambda _ { 1 } \rho \left( \mathbf { X } _ { i } ^ { t } - \mathbf { X } _ { i } ^ { t - 1 } \right) + \lambda _ { 2 } \rho \left( \mathbf { x } _ { i } ^ { t } - \mathbf { x } _ { i } ^ { t - 1 } \right)
\end{align}

式中,$X$ 表示人体的关节点的三维坐标,$X_i^t$ 表示在第$t$时刻人体的第$i$个关节的三维空间位置。同样,我们可以将人体的三维关节点坐标投影到像素坐标系中,得到人体的二维关节点位置,我们希望在像素坐标系中,人体的运动同样也是平滑的,因此式中$x_{ic}^t$ 表示在$t$时刻人体的第$i$个关节在第$c$个相机的视角下的二维关节点位置。

对于人体来说,我们知道有许多姿态都是不合理的,也就是说,我们可以对人体的姿态分布有一个先验的知识。而这种先验知识我们可以从大规模的基于标记点的运动捕捉系统中得到。通过基于标记点的运动捕捉系统,我们可以得到许多不同的人的日常动作,从这些动作中我们可以预先学习得到人体的动作分布。\cite{mocap}数据集中包含了大量的人体动作,为了简化模型参数,一般的做法是使用一个混合高斯模型去\cite{我也不知道哪篇}拟合人体的动作数据。混合高斯模型是指\comment{插段定义}。在判断一个姿态是否属于正常的人体姿态时,通常对混合高斯模型构造一个负对数似然函数,以该函数作为损失函数,然后最小化这个函数的值。即
$$
    E _ { J } ( \theta ) = - \log \left( \sum _ { i } g _ { i } \mathcal { N } \left( \theta ^ { t } ; \mu _ { i } , \Sigma _ { i } \right) \right)
$$
式中,$\mu_i, \Sigma_i$ 分别表示第$i$个高斯分布的均值和协方差,$g_i$ 表示第$i$个高斯分布的权重。通过这种方式,可以表达分布差异较大的人体姿态,并且这些姿态都是属于人体可能存在的姿态

\section{轮廓与纹理重建}

在上一章的内容中,我们完成了使用一个参数驱动的模型拟合到重建出的三维关键点上。使用这种模型表示能够得到一个各个关节具有六自由度的人体模型,但是其仍存在许多不足。例如,这中参数化的人体模型表示的是一个泛化的人体模型,而没有针对单个人的细致的纹理。由于我们的整个方法都是基于RGB图片的,因此对人体的外观没有限制,因此我们可以在同样的实验环境下同时得到人的轮廓信息,再使用人体的轮廓信息对我们得到的人体网格模型进行优化。这一部分内容中,我们首先介绍一下从图片中获得人体轮廓的方法,接着将其运用到我们所录制的视频数据上,获得人体的轮廓信息,接着根据人体的轮廓信息进行对前一步得到的网格进行变形,最后得到人体的带纹理的模型的结果。

\subsection{自动分割}
从图像中获得人体的轮廓也是计算机视觉中的一个重要的研究方向,这个问题指的是从一张包含了一个或者多个人体的图片中,去判断一个像素是属于人体的像素还是不属于人体的像素;在这基础上还拓展出了实例分割任务,即是说判断一个像素是否属于人体身上的像素的同时,还需要将不同人的像素区别开,能够识别出图像中不同的人体;另一方面,对于单个人体,在判断一个像素是否属于该人的基础上,还可以判断这个像素属于人体的哪一个部位,这种任务一般叫做语义分割任务。这个任务在对图片或视频中的人体进行分析中是十分基础的,同时也是非常关键的。随着深度学习的发展,尤其是全卷积网络(FCN)的出现,人体的语义分割任务取得了较大的进步。

Gong等人\cite{cihppgn}提出了一种不需要检测的部件分组网络(Detection-free Part Grouping Network,以下简称PGN)。这个网络主要分为两个部分,第一个部分是进行语义分割,即能够区别人体的不通部位,第二个部分是进行边缘检测,主要的目的是将人体的边缘提取出来。在这两个部分的后面,需要将两个部分进行融合,将分割的结果进行优化,最后再对每个部分的人体的标记进行提取。

\subsubsection{网络结构}

由于我们在进行重建时,目前只考虑人体的整体的轮廓,因此我们对语义分割的结果进行二值化处理,得到属于人体的轮廓信息,如图\ref{fig:mask}所示,从图中可以看出,我们的轮廓提取的模型能够在该场景下得到较为准确的分割结果。

\subsection{轮廓重建}
根据前面得到的每个视角下的人体轮廓信息,以及之前的标定的相机参数,传统的对人体进行重建的方法是使用可视外壳(Visual Hull)的方法,该方法的思路是在空间中维护一个\(N\times N\times N\)的三维网格结构信息,将各个试图的轮廓反投影到该三维网格之中,再通过Marching Cube的算法来获得人体的三维结构。该方法的缺点是,在进行恢复的时候体素占据了大量的计算机存储空间,同时该方法的进度是由体素的分辨率决定的,而通常该分辨率都无法设置到很大,因此该方法的分辨率也较低。并且使用该方法的时候没有预先知道想要恢复的结构的大小,因此使用体素来进行融合会带来许多误差,使得体素的网格无法很好地融合各个视角的信息。

而在我们这个问题中,由于我们已经预先使用了一个参数化人体几何模型来对人体进行了表示,相当于我们的人体模型以及获得了一个较为粗略的表示。我们就可以利用这种先验的几何信息来对问题进行约束,而不是像使用可视外壳的方法没有任何先验信息地进行重建。同样,在考虑人体几何模型的条件下,依然有两种方法可以选择:一种是将各个视角的轮廓信息反投影到三维世界坐标中;另一种是将人体的几何模型投影到各个相机平面上。在将各个视角的轮廓信息反投影到三维世界坐标时,我们需要考虑如何对这些三维结构信息进行表示。常用的做法是使用体素的方法,即将一定范围内的有限大小的空间划分成一个\(N\times N \times N\)的网格结构。但是这样的做法将会降低模型的进度,因为进行了离散化处理,那么可能精细的轮廓信息就无法充分利用。因此我们采用了另一种途径,即将几何模型投影到各个视角的相机上,得到一个当前的各个视角的几何模型的轮廓信息,将这个轮廓信息与我们直接从图片中得到的轮廓信息进行对比,使用他们之间的差异作为损失函数,来行成一个新的优化问题,通过对这个优化问题进行求解,即可将我们的几何模型优化到一个更加贴合我们的模型的状态。

问题的关键在于,一般难以直接使用投影到图像上的轮廓作为损失函数,因为将几何模型投影到图像上转化到相机坐标系的过程会有离散化的操作,而标准的离散化的操作过程在大部分地方的导数都为0。这就导致了如果一个损失函数中包含了离散化的操作,那么整个函数的导数将几乎处处为0,这也就使得优化过程无法进行。因此最近几年的一些工作中,有许多人研究如何将离散化这一过程变得更加连续。在Kato\cite{neuralrender}等人的2017年的工作中,他们将离散化的过程进行了修改。例如对于图\ref{fig:nmr}(a)中的一个三角形面片,将其投影到像素平面上时,对于一个点来说,以输出的是轮廓为例。通常在输出轮廓时,用1来表示该像素属于人体,用0来表示该像素属于背景部分。那么标准的离散化的过程如图\ref{fig:nmr}(b)所示,如果该像素位于三角形的内部,那么其像素值即为1,如果该像素不位于该三角形的内部,那么其像素值就为0。对于这样的标准的离散化过程,其导数会处处为0,如图\ref{fig:nmr}(c)所示。虽然这个函数在边缘处的导数为无穷大,但是由于像素值是不连续的,因此通常边缘处的导数都无法取到,因此像素值关于像素位置的导数就全部是0了,导致没有梯度可以反向传递回去,因此无法直接使用该方法进行梯度下降优化。因此Kato\cite{neuralrender}等人对\ref{fig:nmr}(b)中的函数进行了修改,这样修改使得在边缘处不是直接从0突变到1,而是有一定斜率的变化。这样的离散化方法如果输出图片的话,就相当于在边缘处进行了模糊。这样设计的好处就是,该函数的梯度不再是处处为0了,而是会分段的有值,如\ref{fig:nmr}(e)所示。 
\begin{figure}[htbp]
    \centering
    \includegraphics[width=0.5\linewidth]{figure/mask/neuralrender}
    \caption{\label{fig:nrm} 渲染离散化过程示意}
\end{figure}

我们对我们前一步估计得到的人体几何模型进行投影,得到其在每个相机上的轮廓视图,同时又使用了PGN的方法直接从图片中提取得到轮廓,我们将两个轮廓叠在一起进行对比,如图\ref{fig:maskcom}所示。图中蓝色部分表示两个轮廓的重合部分,红色部分表示实际轮廓中多出的部分,青色部分表示人体几何模型多出的部分。从图\ref{fig:maskcom}中可以看出,我们之前重建的结果较为可靠,直接使用人体几何模型来表示就可以在大部分区域与实际中的人体轮廓。为了对该过程进行量化,我们使用重叠度(Intersection Over Union,简称IoU)这一指标来进行衡量。重叠度通常用来进行评价目标检测任务,其定义是两块区域的交集与两块区域的并集的比例,在我们的问题中,我们使用一个二值化的矩阵\(S\)来表示轮廓,即
\begin{equation}
    S_{ij} = \begin{cases}
        1, & \text{\((i,j)\)处像素属于人体区域} \\
        0, & \text{\((i,j)\)处像素不属于人体区域}
    \end{cases}
\end{equation}
对于两个轮廓,其重叠度计算公式则为
\begin{equation}
    IoU = \frac{S_a \cap S_b}{S_a \cup S_b} = \frac{S_a \cdot S_b}{S_a + S_b - \cdot S_a\cdot S_b}
\end{equation}

为了使我们的模型更加贴合实际的人体轮廓,如果只使用参数化模型会无法拟合到各个轮廓上,因此我们对之前的人体三维网格模型进行修改,对于模型上的每一个顶点,我们都给其加上一个小的偏移量\(d_i\),通过这个偏移量去控制人体模型,使其更加接近实际轮廓。同样我们通过定义损失函数,然后再去最小化这个损失函数的方法,损失函数也由不同的项组成。首先是模型拟合项
\begin{equation}
    L_{silh} = || S_{real} - \Pi(V(\theta, \beta, D)) ||_2
\end{equation}


\begin{figure}[htbp]
    \centering
    \includegraphics[width=0.5\linewidth]{figure/result/mask/shuai}
    \caption{\label{fig:maskcom} 重建的轮廓与图片中获得的轮廓对比}
\end{figure}

\subsection{动作迁移}
在影视制作业中,通常会有这样的需求,即通过专业的动作演员生成一段特殊的动作,例如高难度的武打动作,然后将这段动作序列迁移到另一个演员模型上,或者迁移到一个动画模型上。在我们的算法中,由于使用了人体的几何模型,因此我们同样也可以完成这样的事情。其主要流程为:通过我们的动作捕捉系统获得演员的动作序列;通过其他方式获得演员的模型贴图,或者同样通过我们的系统获得演员的模型贴图;将动作序列与模型贴图进行组合,即可得到另一个演员的完成这个动作的图片或者视频了。

例如针对我们获取的演员帅的动作数据,我们将重建获得的人体几何模型的姿态参数和形状参数以及在世界坐标系中的位置的数据保存下来。我们将其进行纹理贴图,纹理文件来源于 等人的论文所提供的纹理贴图\cite{surreal}。为了体现模型的通用性,我们

\unsure{}{这里写如何把别人的纹理贴过来}

\unsure{}{这里写如何贴纹理}

\section{人体三维重建工具}

\subsection{总体框架}
整合前面的所有二维关键点估计部分、三维关键点重建、人体几何模型重建部分,我们设计了一个多视角的单目视频人体重建工具。该工具的设计主要流程如下所示。
\unsure{}{插个流程图}

使用者只需要读入多个视角的图像或视频数据,以及摄像机的内部参数和外部参数,就可以对其中的人体动作进行捕捉,并且得到一个人体的三维模型,以及其动作。

\subsection{界面设计}
在界面设计的时候,我们主要秉承着简洁易用的原则,并且能够对我们的结果进行较好的可视化。我们设计的系统的用户交互界面设计如图 所示。该工具包含了人体三维重建的各个步骤,如相机标定、二维关键点估计、三维关键点重建、人体几何模型重建等。

\subsection{使用说明}
在使用时,用户需要依次选择功能。如果相机没有进行标定,需要先选择各个相机标定图片文件所在的文件夹,对其进行标定,得到相机标定的参数。如果相机已经标定,那么需要导入相机参数所在的路径。载入相机参数后,可以选择不同的二维关键点检测方式,进行检测,接着对结果进行可视化。完成二维关键点检测之后,即可运行三维关键点重建,同样提供了两种方式可以选择。最后获得三维关键点之后即可运行人体几何模型重建。





\section{视频中的人体重建}

\section{结论与展望}
\subsection{结论}
在本研究中,我们完成了从光场系统获得多个视角下的视频序列采集,首先通过深度卷积神经网络得到人体二维关节点坐标;再通过Ceres求解无约束条件下的最小二乘问题,得到了人体三维关节点坐标;在此基础上,我们引入了参数化的人体模型SMPL,通过求解优化问题恢复出该模型的参数,再结合几何信息得到了更细致的模型表达,完成了对人体模型的三维重建过程。最后,我们使用捕获到的视频中的图像信息,对人体表面的纹理进行了恢复,得到更具表现能力的模型。

\subsection{论文中出现的问题及思考}
论文里主要完成了从多视角的单目视频中采集得到的人体动作视频序列中恢复出人体模型的过程,在完成该任务的过程中,虽然基本完成了任务需求,但是仍然存在一些问题,主要有以下几点:
\begin{enumerate}
    \item 三维关键点重建的过程中,
    \item 在进行人体模型的参数拟合的时候,速度在1-2s每帧,仍然无法达到实时的效果。因此下一步考虑将这一步使用C++重新实现,提高计算速度,达到实时的效果
    \item 在进行人体模型的轮廓拟合的时候,由于将人体模型渲染这一步较慢,而在优化的过程中又需要不断迭代渲染,因此这一步也比较慢,要完成一个时刻23个相机的轮廓优化,需要2分钟以上才能收敛.
    \item 在进行人体模型的纹理贴图的时候,同样渲染的过程较慢,并且相比轮廓渲染,这一步还需要使用纹理信息,因此渲染会更慢,同时比较占显存,完成一个时刻的纹理贴图同样也需要5分钟以上的时间才能收敛。
\end{enumerate}


\subsection{展望}
在完成毕业设计的过程中,我认识到了使用GPU对计算速度的提升,在我们的任务中,虽然目前也仍然较慢,但是如果完全不使用GPU进行并行计算加速的话,我们的任务需要的时间将是一个天文数字。使用GPU进行加速之后使得我们的许多任务成为了可能,因此在将来的学习中需要进一步加强并行计算的能力。另一方面,我们可以看到使用高清相机在动作捕捉领域的巨大前景,以及其具有的应用潜力。目前我们使用的图像信息只有从图像中获得的人体轮廓,以及在贴图的时候使用到的图像颜色信息。实际上对于图像信息我们还可以有许多应用方式,例如从图像中感知到环境中的其他物体与环境的地理信息,从而理解人与环境中的其他物体的交互方式,从图像中得到人受到的周围环境的约束信息,而这是其他方式的动作捕捉设备无法保留的信息。另一方面,单从图形学的角度,从图像中还可以获得环境的光照、阴影信息,从这些信息着手也可以对人体进行更好的重建,使得最后渲染出来的结果更加逼真,更加接近真实环境中的光照条件、反射条件。

另一方面,我们也不一定将信息局限于视频信息,在这套系统中,我们可以在不改变原有架构的基础上增加一些其他信息。例如可以在人体身上佩戴一些较小的惯性传感器,获取人体的不同部位的运动信息,与从视频中获得的信息结合,从而更好的完成对人体的姿势的估计。

% \unsure{}{这里写的时候写一下结合更多信息的,以及RGB的优势}
