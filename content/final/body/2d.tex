\section{人体二维姿态估计问题的研究}
\subsection{问题描述}
人体二维人体姿态估计问题,即是说给定输入的图片或视频序列,进行处理得到相应的人体各个关节在图片上的位置。
\subsection{研究方法比较}
人体二维姿态估计目前有两个主流方案,第一种是自顶向下的方法,这种方法先检测图片中的人体,得到一个人体的检测框,然后去分别检测每一个人的人体区域中的姿态;另一种是自底向上的方法,首先检测出图片中的所有的人的肢体的关节节点,然后将关节节点进行拼接,得到人体的骨架。自顶向下的方法中,姿态检测的准确度会依赖于人体区域框的检测质量;而在自底向上的方法中,如果两个人离得较近,会出现拼接错误的情况;同时由于依赖的是关节之间的联系,所以对全局信息的获取会有不足。目前主流的开源人体二维检测方案主要有几种,罗列如下。
\begin{itemize}
    \item CPN 
    \item OpenPose
    \item AlphaPose
\end{itemize}
下面依次介绍这几种模型。
\subsubsection{OpenPose}
OpenPose是由CMU提出的多人二维关键点实时检测方案,该方案是基于自底向上的思想的。该方法使用了一种叫部件亲和场(PAF)的非参数表示方法,通过这种方法将检测出来的人体关节组合起来。该方法具有较高的精度以及较好的实时性。该方法可以同时对人体的身体部分、双脚、手部关节点、脸部关键点进行检测,输出结果较为丰富。

\subsubsection{AlphaPose}
AlphaPose使用的是自顶向下的方法,

\comment{这里应该有个结果比较}

\subsection{应用}
\comment{这里主要写将模型拿来应用到我们的数据上,以及结果}
