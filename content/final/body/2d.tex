\section{人体二维姿态估计问题的研究}
\subsection{问题描述}
人体二维人体姿态估计问题,即是说给定输入的图片或视频序列,进行处理得到相应的人体各个关节在图片上的位置。
\subsection{研究方法比较}
人体二维姿态估计目前有两个主流方案,第一种是自顶向下的方法,这种方法先检测图片中的人体,得到一个人体的检测框,然后去分别检测每一个人的人体区域中的姿态;另一种是自底向上的方法,首先检测出图片中的所有的人的肢体的关节节点,然后将关节节点进行拼接,得到人体的骨架。自顶向下的方法中,姿态检测的准确度会依赖于人体区域框的检测质量;而在自底向上的方法中,如果两个人离得较近,会出现拼接错误的情况;同时由于依赖的是关节之间的联系,所以对全局信息的获取会有不足。目前主流的开源人体二维检测方案主要有几种,罗列如下。
\begin{itemize}
    \item CPN 
    \item OpenPose
    \item AlphaPose
\end{itemize}
下面依次介绍这几种模型。
\subsubsection{OpenPose}
OpenPose是由CMU提出的多人二维关键点实时检测方案,该方案是基于自底向上的思想的。该方法使用了一种叫部件亲和场(Part Affinity Fields,简称PAF)的非参数表示方法,通过这种方法将检测出来的人体关节组合起来。部件亲和场的表示方法是指,对于人体上的骨架上的每一个像素,去回归这一段骨架的方向向量。该方法具有较高的精度以及较好的实时性。该方法可以同时对人体的身体部分、双脚、手部关节点、脸部关键点进行检测,输出结果较为丰富。
\begin{figure}[H]
    \centering
    \includegraphics[width=\linewidth]{figure/2dpose/openpose}
    \caption{\label{fig:2d-op} OpenPose流程图}
\end{figure}
该方法主要分为几个步骤,首先对于一张输入的图片,OpenPose将整张图片都输入给深度卷积神经网络,通过网络同时去回归人体的独立的关节点的位置,以及部件亲和场。通过对回归的结果进行组合将候选的关节进行组合,匹配属于同一个人的关节,最后就能得到整个人体的完整的身体姿态。

\subsubsection{AlphaPose}
AlphaPose是上海交通大学卢策吾教授团队开源的二维人体关键点检测方法,该方法使用的是自顶向下的方法,即首先检测人体所在的框的区域,再对框中的人体进行关键点检测。首先通过区域候选网络(Region Proposal Network)提出候选的人体区域,然后将所得到的所有的候选的人体区域框分为两支,一支通过单人姿态估计(Single Person Pose Estimation)算法,以及姿态非极大值抑制算法得到姿态,另一支直接通过单人姿态估计得到人体的姿态,最后将两支分路的结果进行整合,得到最终的人体姿态。 
\begin{figure}[H]
    \centering
    \includegraphics[width=\linewidth]{figure/2dpose/alphapose}
    \caption{\label{fig:2d-ap} AlpahPose流程图}
\end{figure}

\subsubsection{CPN}
级联金字塔网络(Cascaded Pyramid Network, 简称CPN)是有旷视科技提出的模型,该模型取得了COCO2017人体关键点挑战赛的冠军,该模型主要面向的问题是单张图片中的多人姿态估计问题。该模型属于自顶向下的方法,即首先生成人体的边界框,之后对单个人体的框进行关键点检测。级联金字塔网络包含了两个阶段,第一个阶段为全局网络(GlobalNet),这部分用来从图像中提取特征,可以定位特征较为明显的关键点,例如手和脚,但是无法直接从图片中识别出被遮挡的关键点;第二个阶段为精炼网络(RefineNet),这一阶段主要通过整合全局网络中识别得到的图片特征,来推断被遮挡住的关键点的位置。即是说理解可以看到的关键点所提供的上下文信息,完成所有关键点的检测任务。
\begin{figure}[H]
    \centering
    \includegraphics[width=.4\linewidth]{figure/2dpose/cpn}
    \caption{\label{fig:2d-cpn} CPN流程图}
\end{figure}
论文中的全局网络的骨架是深度残差网络\cite{resnet},通过使用U形结构来整合不通尺度的空间分辨率的特征以及语义信息。精炼网络部分中,为了提高计算的效率,同时保持信息在传输过程中的完整性,精炼网络部分首先讲不同尺度的特征图进行堆叠,然后通过上采样和融合,将不同层次的信息进行融合。

\subsection{模型比较}
为了验证上述二维关节点检测模型在我们的实验环境下的准确度,我们需要对其进行定量的分析。由于我们的实验设备没有在人身上贴标签,因此无法得到准确的三维关键点以及二维关节点的坐标,因此我们选择了广泛使用的Human3.6M数据集。该数据集是最大的人体三维姿态数据集之一,包含了360万的图片,其中包括了11个演员以及15个日常动作,例如吃饭,走路等。该数据集的采集环境为室内环境,并且使用了总共4个相机来进行数据采集。该数据集通过在人体贴上标志物来获得人体的三维空间的关节位置,相对于从图片直接估计来说较为准确,因此我们通过该数据集来对我们的算法进行定量分析。

在这一部分中,我们使用Human3.6M数据集的图片,通过三种不同的人体二维关节点检测方法获得其关节位置,与其提供的二维关节点的位置进行比较,计算三种方法的误差。由于二维关节点的训练数据集中,关节的位置通常都是由人类标注者进行标注,而Human3.6M的关节点的位置是由其贴的标志的位置决定的,因此两者的关节定义有所不同,而在三种二维关节点检测方法中,其各自使用的训练数据集也有所区别,因此我们只选择了人的身体上的主要的几个点进行比较。

\begin{figure}[H]
    \centering
    \includegraphics[width=.4\linewidth]{figure/2dpose/compare}
    \caption{\label{fig:2d-compare} 三种二维人体关节点检测方法结果比较}
\end{figure}

\begin{figure}[H]
    \centering
    \includegraphics[width=.4\linewidth]{figure/2dpose/compare}
    \caption{\label{fig:2d-loss} 各个关节的误差分布对比}
\end{figure}

\begin{table}[H]
    \centering
    \begin{tabular}{lccc}
        \hline
         Name       &   OpenPose &     CPN &   AlphaPose \\
        \hline
         \text{误差均值(像素)}   &    9.62135 & 9.32769 &     8.78416 \\
         \text{误差标准差} &    6.35488 & 7.85625 &     4.63783 \\
        \hline
    \end{tabular}
    \label{tab:plan}%
    \caption{三种二维检测人体方法的误差}
\end{table}


\subsection{应用}
基于以上比较,我们发现\comment{模型}的精度较高,因此我们选择使用该模型在我们的模型上应用。部分图片的二维人体关节点检测结果如图所示。从二维关节点检测结果可以看出,目前的二维人体关节点模型可以在我们的数据上取得较好的结果。基于这一部分的结果,我们才能对人体的三维关节点进行重建。

