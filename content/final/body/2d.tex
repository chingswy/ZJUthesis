\section{人体二维姿态估计问题的研究}
\subsection{问题描述}
人体二维人体姿态估计问题,即是说给定输入的图片或视频序列,进行处理得到相应的人体各个关节在图片上的位置。
\subsection{研究方法比较}
人体二维姿态估计目前有两个主流方案,第一种是自顶向下的方法,这种方法先检测图片中的人体,得到一个人体的检测框,然后去分别检测每一个人的人体区域中的姿态;另一种是自底向上的方法,首先检测出图片中的所有的人的肢体的关节节点,然后将关节节点进行拼接,得到人体的骨架。自顶向下的方法中,姿态检测的准确度会依赖于人体区域框的检测质量;而在自底向上的方法中,如果两个人离得较近,会出现拼接错误的情况;同时由于依赖的是关节之间的联系,所以对全局信息的获取会有不足。目前主流的开源人体二维检测方案主要有几种,罗列如下。
\begin{itemize}
    \item CPN 
    \item OpenPose
    \item AlphaPose
\end{itemize}
下面依次介绍这几种模型。
\subsubsection{OpenPose}
OpenPose是由CMU提出的多人二维关键点实时检测方案,该方案是基于自底向上的思想的。该方法使用了一种叫部件亲和场(PAF)的非参数表示方法,通过这种方法将检测出来的人体关节组合起来。该方法具有较高的精度以及较好的实时性。该方法可以同时对人体的身体部分、双脚、手部关节点、脸部关键点进行检测,输出结果较为丰富。
\begin{figure}[H]
    \centering
    \includegraphics[width=.4\linewidth]{figure/2dpose/openpose}
    \caption{\label{fig:2d-op} OpenPose流程图}
\end{figure}

\subsubsection{AlphaPose}
AlphaPose使用的是自顶向下的方法,
\begin{figure}[H]
    \centering
    \includegraphics[width=.4\linewidth]{figure/2dpose/alphapose}
    \caption{\label{fig:2d-ap} AlpahPose流程图}
\end{figure}

\subsubsection{CPN}
\begin{figure}[H]
    \centering
    \includegraphics[width=.4\linewidth]{figure/2dpose/cpn}
    \caption{\label{fig:2d-cpn} CPN流程图}
\end{figure}

\subsection{模型比较}
为了验证上述二维关节点检测模型在我们的实验环境下的准确度,我们需要对其进行定量的分析。由于我们的实验设备没有在人身上贴标签,因此无法得到准确的三维关键点以及二维关节点的坐标,因此我们选择了广泛使用的Human3.6M数据集。该数据集是最大的人体三维姿态数据集之一,包含了360万的图片,其中包括了11个演员以及15个日常动作,例如吃饭,走路等。该数据集的采集环境为室内环境,并且使用了总共4个相机来进行数据采集。该数据集通过在人体贴上标志物来获得人体的三维空间的关节位置,相对于从图片直接估计来说较为准确,因此我们通过该数据集来对我们的算法进行定量分析。

在这一部分中,我们使用Human3.6M数据集的图片,通过三种不同的人体二维关节点检测方法获得其关节位置,与其提供的二维关节点的位置进行比较,计算三种方法的误差。由于二维关节点的训练数据集中,关节的位置通常都是由人类标注者进行标注,而Human3.6M的关节点的位置是由其贴的标志的位置决定的,因此两者的关节定义有所不同,而在三种二维关节点检测方法中,其各自使用的训练数据集也有所区别,因此我们只选择了人的身体上的主要的几个点进行比较。

\begin{figure}[H]
    \centering
    \includegraphics[width=.4\linewidth]{figure/2dpose/compare}
    \caption{\label{fig:2d-compare} 三种二维人体关节点检测方法结果比较}
\end{figure}

\begin{figure}[H]
    \centering
    \includegraphics[width=.4\linewidth]{figure/2dpose/compare}
    \caption{\label{fig:2d-loss} 各个关节的误差分布对比}
\end{figure}


\subsection{应用}
\comment{这里主要写将模型拿来应用到我们的数据上,以及结果}
