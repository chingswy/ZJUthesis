\cleardoublestylepage{common}
\newcommand{\sigp}{$\sigma$点}
\newcommand{\sigm}{\sigma^{(m)}}
\section{模型建立}
\subsection{无迹卡尔曼滤波模型}
\subsubsection{卡尔曼滤波}
对于一个零输入的线性系统,其系统的动力学方程可以写为
\begin{align}
    x_k &= A_kx_{k-1} + \omega_k \\
    y_k &= C_kx_k + v_k
\end{align}
这里的$\omega_k$表示状态噪声,$v_k$表示观测噪声,假设这里的噪声都是服从零均值的高斯分布,即
\begin{equation}
    \omega_k \sim N(0,R),v_k\sim N(0,Q)
\end{equation}
那么经典的卡尔曼滤波可以分为先验估计和后验更新两个步骤:

1. 先验估计:
\begin{equation}
    \bar{x}_k = A_k\hat{x}_{k-1} + u_k,\bar{P}_k = A_k\hat{P}_{k-1}A_k^T + R.
\end{equation}

2. 更新:先计算卡尔曼增益$K$
\begin{align}
    P_{xy} &= \bar{P}_kC_k^T \\
    P_{yy} &= C_k\bar{P}_kC_k^T + Q_k \\
    K &= P_{xy}P_{yy}^{-1} = \bar{P}_kC_k^T(C_k\bar{P}_kC_k^T + Q_k)^{-1}
\end{align}
然后计算后验概率分布
\begin{align}
    \hat{x}_k & = \bar{x}_k + K(y_k-C_k\bar{x}_k) \\
    \hat{P}_k &= (I-KC_k)\bar{P}_k.
\end{align}

\subsection{无迹卡尔曼滤波}
在卡尔曼滤波中,假设系统是线性的。但是在实际问题中,运动方程和观测方程中出现了非线性变换,因此系统成为了非线性系统。高斯分布通过非线性系统将不一定仍保持高斯分布。因此在描述人体状态时常常不满足卡尔曼滤波的假设,为了仍然使用卡尔曼滤波,我们将采用改进之后的卡尔曼滤波器。

无迹卡尔曼滤波是通过选择$\sigma$点来描述状态变量的均值和方差。即设状态变量为$x$,$\bar{x},P$分别表示状态变量的均值和方差,那么计算\sigp,
\begin{align}
    \sigma^{(0)} &= \bar{x} \\
    \sigma^{(m)} &= \bar{x} \pm \left(\sqrt{(M+\lambda)P}\right)_m,m=1,\ldots,2M,
\end{align}
其中$(\sqrt{\cdot})_m$表示Cholesky分解的第$m$列,$\lambda$是控制\sigp 和均值点的距离的参数。那么$f(x)$的均值和方差就可以计算得到:
\begin{align}
    E\left[f(x)\right] &\approx \mu = \sum_{m=0}^{2M}w_m f(\sigm),\\
    cov\left[f(x)\right] &\approx \sum_{m=0}^{2M}w_m (f(\sigm) - \mu)(f(\sigm) - \mu)^T
\end{align}
其中权重的定义为
\begin{align}
    w_0 &= \frac{\lambda}{\lambda+M},\\
    w_m &= \frac{1}{2(\lambda + M)},m = 1,\ldots,2M.
\end{align}
此时计算卡尔曼增益与经典的卡尔曼滤波有所区别,即
\begin{align}
    P_{yy} \approx \sum_{m=0}^{2M}w_m(h(\sigm) - \hat{y}_t)(h(\sigm) - \hat{y}_t)^T \\
    P_{xy} \approx \sum_{m=0}^{2M}w_m(f(\sigm) - \hat{y}_t)(h(\sigm) - \hat{y}_t)^T 
\end{align}

\subsection{人体骨架模型}
对于一个人体骨架,我们使用$p$个关节点来表示,那么$t$时刻人体的三维姿态就可以通过这$p$个关节点的三维位置来表示,记为$S_t \in \mathbb{R}^{3\times p}$\comment{按MonoCap里面的记号写的}。我们假设人体的骨架长度是已知的,那么我们可以使用每一段骨架的朝向来描述人体的姿态。即,对于同一段骨架的两端的关节点的坐标$x_i(t),x_j(t)$,其骨架长度约束可以描述为
\begin{equation}
    ||x_i(t) - x_j(t)||_2 = l_{ij} = const.
\end{equation}
那么其朝向可以记为
\begin{equation}
    x_{ij} \doteq \frac{x_j-x_i}{||x_i(t) - x_j(t)||_2} \in \mathbb{S}^2
\end{equation}
因此,一个子节点的位置可以通过其父节点的位置,以及这一段骨架的朝向来唯一确定。同时,我们考虑其运动速度,那么则有
\begin{equation}
    \dot{x}_j(t) = \dot{x}_i(t) + l_{ij}\omega_{ij}(t) 
\end{equation}
其中$\omega_{ij} \doteq \dot{x}_{12}(t)$。因此,子节点的速度可以通过角速度以及父节点的速度唯一确定。其中$\omega_{ij}(t)\in T_{x_{ij}(t)}\mathbb{S}^2$。

\comment{差一个建立坐标的图}

\subsubsection{运动模型}
\comment{这里写在manifold上如何进行运动的}
\subsubsection{观测模型}
\comment{写相机观测这些}

\subsection{人体骨架上的卡尔曼滤波}


% \section{正文}

% \subsection{节标题}

% \subsubsection{小节标题}

% \par 如\autoref{fig:a-figure}所示,这是一张由TIKZ\autocite{tikz}包画出的图片,我们也可以用includegraphics来插入现有的jpg等格式的图片,如\autoref{fig:zju-logo}。

% \begin{figure}[ht]
%     \centering
%     \begin{tikzpicture}[thick,help lines/.style={thin,draw=black!50}]
%         \def\A{\textcolor{input}{$A$}} \def\B{\textcolor{input}{$B$}}
%         \def\C{\textcolor{output}{$C$}} \def\D{$D$}
%         \def\E{$E$}

%         \colorlet{input}{blue!80!black} \colorlet{output}{red!70!black}
%         \colorlet{triangle}{orange}

%         \coordinate [label=left:\A] (A) at ($ (0,0) + .1*(rand,rand) $);
%         \coordinate [label=right:\B] (B) at ($ (1.25,0.25) + .1*(rand,rand) $);

%         \draw [input] (A) -- (B);

%         \node [name path=D,help lines,draw,label=left:\D] (D) at (A) [circle through=(B)] {};
%         \node [name path=E,help lines,draw,label=right:\E] (E) at (B) [circle through=(A)] {};

%         \path [name intersections={of=D and E,by={[label=above:\C]C}}];

%         \draw [output] (A) -- (C) -- (B);

%         \foreach \point in {A,B,C}
%             \fill [black,opacity=.5] (\point) circle (2pt);

%         \begin{pgfonlayer}{background}
%             \fill[triangle!80] (A) -- (C) -- (B) -- cycle;
%         \end{pgfonlayer}
%     \end{tikzpicture}

%     \caption{\label{fig:a-figure}这是一张图片}
% \end{figure}

% \begin{figure}[ht]
%     \centering
%     \includegraphics[width=.4\linewidth]{logo/zju}
%     \caption{\label{fig:zju-logo}浙江大学LOGO}
% \end{figure}

% \par 如\autoref{tab:sample}所示,这是一张自动调节列宽的表格。

% \begin{table}[ht]
%     \caption{\label{tab:sample}自动调节列宽的表格}
%     \begin{tabularx}{\linewidth}{c|X<{\centering}}
%         \hline
%         第一列 & 第二列 \\ \hline
%         xxx & xxx \\ \hline
%         xxx & xxx \\ \hline
%         xxx & xxx \\ \hline
%     \end{tabularx}
% \end{table}

% \par 如\autoref{equ:sample},这是一个公式

% \begin{equation}
%     \label{equ:sample}
%     A=\overbrace{(a+b+c)+\underbrace{i(d+e+f)}_{\text{虚数}}}^{\text{复数}}
% \end{equation}
