\section{人体几何模型的应用}
\subsection{问题描述}
在上一部分中,我们已经从图片中得到人体的三维关节点的坐标,但是在实际应用中,简单的骨架表示不够具有表现力,因此最新的论文里的做法是使用一个预先生成好的参数化的人体几何模型,通过优化算法,或者直接通过网络估计,得到该模型的参数。在我们的问题中,希望通过输入一个人体的骨架的三维坐标,输出该参数化模型的参数。由于骨架坐标无法完全确定该模型的参数,因此我们需要引入一些额外的约束,来完成该求解过程。

\subsection{模型介绍}
\comment{这里写SMPL的模型介绍,找一篇论文对照着写一下}
SMPL模型是一个高效的线性人体模型,模型总共定义了\(N = 6890\)个顶点,作为人体的模板\(\bar{\bm{T}}\)。为了使该模型能够描述不同的人体姿态,该模型定义了人体的24个关节,每个关节具有三个旋转的自由度(包括整个身体的旋转),再加上身体的平移的三个自由度,该模型使用了\(3\times 24 + 3 = 75\)个位姿参数\(\theta\)来描述人体的姿态。为了能够描述不同人体的形状的差异,模型使用了形状参数\(\beta\)来对人体模板的点进行非刚性变形。也就是说,人体模板首先会根据形状参数和姿态参数进行变形:
\begin{equation}
    T(\beta, \theta) = \bar{\bm{T}} + B_s
\end{equation}

\subsection{算法分析}
\comment{这里写如何Fit,就是简单的loss函数}