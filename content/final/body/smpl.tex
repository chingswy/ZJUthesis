\section{人体几何模型的应用}
\subsection{问题描述}
在上一部分中,我们已经从图片中得到人体的三维关节点的坐标,但是在实际应用中,简单的骨架表示不够具有表现力,因此最新的论文里的做法是使用一个预先生成好的参数化的人体几何模型,通过优化算法,或者直接通过网络估计,得到该模型的参数。在我们的问题中,希望通过输入一个人体的骨架的三维坐标,输出该参数化模型的参数。由于骨架坐标无法完全确定该模型的参数,因此我们需要引入一些额外的约束,来完成该求解过程。

\subsection{模型介绍}
\comment{这里写SMPL的模型介绍,找一篇论文对照着写一下}
SMPL模型是一个高效的线性人体模型,模型总共定义了\(N = 6890\)个顶点,作为人体的模板\(\bar{\bm{T}}\)。为了使该模型能够描述不同的人体姿态,该模型定义了人体的24个关节,每个关节具有三个旋转的自由度(包括整个身体的旋转),再加上身体的平移的三个自由度,该模型使用了\(3\times 24 + 3 = 75\)个位姿参数\(\theta\)来描述人体的姿态。为了能够描述不同人体的形状的差异,模型使用了形状参数\(\beta\)来对人体模板的点进行非刚性变形。也就是说,人体模板首先会根据形状参数和姿态参数进行变形:
\begin{equation}
    T(\beta, \theta) = \bar{\bm{T}} + B_s
\end{equation}

\subsection{算法分析}
\comment{这里写如何Fit,就是简单的loss函数}

我们都知道,人体是具有骨架结构的,也就是说人的运动状态一定是连续的,人的关节坐标无法突变,因此,为了增加结果的鲁棒性,以及重建的人的运动的平滑性,我们通过增加时序误差来对求解结果进行优化。时序误差的定义为

\begin{align} 
    E _ { T } ( \beta , \theta , \mathbf { \Omega } ) = \sum _ { t = 2 } ^ { T } \sum _ { i = 1 } ^ { J } \lambda _ { 1 } \rho \left( \mathbf { X } _ { i } ^ { t } - \mathbf { X } _ { i } ^ { t - 1 } \right) + \lambda _ { 2 } \rho \left( \mathbf { x } _ { i } ^ { t } - \mathbf { x } _ { i } ^ { t - 1 } \right)
\end{align}

式中,$X$ 表示人体的关节点的三维坐标,$X_i^t$ 表示在第$t$时刻人体的第$i$个关节的三维空间位置。同样,我们可以将人体的三维关节点坐标投影到像素坐标系中,得到人体的二维关节点位置,我们希望在像素坐标系中,人体的运动同样也是平滑的,因此式中$x_{ic}^t$ 表示在$t$时刻人体的第$i$个关节在第$c$个相机的视角下的二维关节点位置。

对于人体来说,我们知道有许多姿态都是不合理的,也就是说,我们可以对人体的姿态分布有一个先验的知识。而这种先验知识我们可以从大规模的基于标记点的运动捕捉系统中得到。通过基于标记点的运动捕捉系统,我们可以得到许多不同的人的日常动作,从这些动作中我们可以预先学习得到人体的动作分布。\cite{mocap}数据集中包含了大量的人体动作,为了简化模型参数,一般的做法是使用一个混合高斯模型去\cite{我也不知道哪篇}拟合人体的动作数据。混合高斯模型是指\comment{插段定义}。在判断一个姿态是否属于正常的人体姿态时,通常对混合高斯模型构造一个负对数似然函数,以该函数作为损失函数,然后最小化这个函数的值。即
$$
    E _ { J } ( \theta ) = - \log \left( \sum _ { i } g _ { i } \mathcal { N } \left( \theta ^ { t } ; \mu _ { i } , \Sigma _ { i } \right) \right)
$$
式中,$\mu_i, \Sigma_i$ 分别表示第$i$个高斯分布的均值和协方差,$g_i$ 表示第$i$个高斯分布的权重。通过这种方式,可以表达分布差异较大的人体姿态,并且这些姿态都是属于人体可能存在的姿态