\cleardoublestylepage{common}

\section{绪论}

\subsection{研究背景与意义}
运动捕捉系统发展介绍

传统的光学运作捕捉系统主要是基于标记点的,游戏制作公司和电影公司通常以标记点作为载体。这种方法的好处是精度较高,但是其不足也是很明显的,首先是对拍摄环境要求严格,其次需要被拍摄人穿着紧身服装,以及在服装上贴上标记。此外,这种方式还需要进行许多的手动标记和数据清理的工作,时间成本极高。同时,拍摄的人也无法穿着日常的服装。

在这样的背景下,无标记点(Marker-less)的运动捕捉方式就出现了。这种方式的目标是去除掉基于标记点技术的限制。这种方式的运动捕捉更为灵活,对人体的着装要求不敏感,同时对光照条件也不敏感,因此可以拍摄人们穿着日常服装的数据,并进行处理。


\subsection{国内外研究现状}
pass

\subsection{主要研究内容}
本文旨在提出针对光场系统同步高速采集的视频序列进行自动人体三维重建的软件系统,这个系统涵盖了多视角三维重建的各个阶段。该系统的硬件平台为浙江大学CAD\&CG国家重点实验室所搭建的人体反射场高性能采集系统,以下简称Light Stage系统,通过该平台我们能够采集得到同步的相机阵列视频序列,以采集的数据为基础,我们完成了一套稳定并且鲁棒的自动人体三维重建软件,主要包括了相机标定、颜色校正、人体二维关节点检测、人体三维关节点重建、人体几何模型重建等步骤。不仅能够从单目相机的数据中获得人体的二维关节点位置结果,还能充分利用该平台的多视角的同步特性,综合考虑多视角的几何信息,使得重建的结果更加鲁棒。在此基础上,还对人体几何模型进行了拟合,对人体的各个关节的三自由度进行了估计,有效地捕获了人体的运动过程,能够较好的满足在运动捕捉、电影处理、虚拟现实等相关应用的需求。 