{
    \setlength{\parindent}{0em}
    \par {\zihao{4}\bfseries 一、题目:\Title}
    \\
    \par {\zihao{4}\bfseries 二、指导教师对文献综述、开题报告、外文翻译的具体要求:}

    任务要求:

设计从视频中恢复人体三维姿态的算法,并在任意场景下进行应用,使用Python完成算法编写。

目标:能在摄像头获取的视频上进行算法实施,进行实物模拟
 
进度安排:

2018-11-20 至2018-12-17: 1.查阅相关文献、攔写文献综述  2.分析现有的一两种有代表性的方法
      
2018-12-18 至2019-01-14: 1.继续分析有代表性的现有方法,并将之在本地实现  2.继续阅读论文,并提出改进的方案,并进行理论分析,进行编程实现和实验验证

2019-03-05 至2019-04-01: 继续改进方案,并进行理论分析,同时进行编程实现和实验验证,努力提高模型的性能

2019-04-02 至2019 04 -29: 1.继续改进方案,并进行理论分析,同时进行编程实现和实验验证,努力提高模型的性能  2开始撰写毕业论文

2019-04-30 至2019 05-13: 1.继续改进方案,并进行理论分析,同时进行编程实现和实验验证,努力提高模型的性能  2.完成毕业论文的撰写

}

\mbox{} \vfill

\signaturetime{指导教师(签名)}{2019}{1}{8}
